\chapter{INTRODUCTION TO COMMANDS}
% !TEX root = hazy1.tex

\section{Overview}

This section introduces the commands that drive \Cloudy.
The following
chapters group the commands together by purpose.
Individual commands are
discussed after examples of their use.
This section begins by outlining
conditions that are assumed by default and then goes on to discuss the
various classes of commands (i.e., those that set the incident
radiation field, composition, or the geometry).

Keeping this document parallel with the code is a very high priority.
In case of any confusion, please consult the original source.
The commands are
parsed by the series of routines that have names beginning with
\cdRoutine{``parse''.}
The list of routines can be seen by listing the files \cdFilename{parse*.cpp}.
The second
half of the name indicates the command that is parsed by that routine.

\Cloudy\ is designed so that a reasonable set of
initial conditions to be assumed by default so that
a minimum number of commands are needed to drive it.
These default conditions are summarized
in Table \ref{tab:DefaultConditions},
which also lists the
commands that change each assumption.

\begin{table}[t]
\centering
\caption{\label{tab:DefaultConditions}Default Conditions}
\begin{tabular}{l l l }
\hline
Quantity& Value& Command\\
\hline
Default inner radius& $10^{30}$ cm& \cdCommand{radius}\\
Default outer radius& $10^{31}$ cm& \cdCommand{radius}\\
Highest allowed temperature& \TEMPLIMITHIGH\\
Stop calculation when
$T$ falls below this value& \TEMPSTOPDEFAULT& \cdCommand{stop temperature}\\
Relative error in
heating-cooling balance& 0.005& \cdCommand{set temperature convergence}\\
Relative error
in electron density& 0.0& \cdCommand{set eden convergence}\\
Relative intensity of faintest
line to print& $10^{-3}$& \cdCommand{print line faint}\\
Low-energy limit to continuum& \emm\\
High-energy limit to continuum& \egamry\\
Limiting number of zones& 1400& \cdCommand{set nend}\\
Total hydrogen column density& $10^{30}$ cm$^{-2}$& \cdCommand{stop column density}\\
\hplus\ column density& $10^{30}$ cm$^{-2}$& \cdCommand{stop column density}\\
\hO\ column density& $10^{30}$ cm$^{-2}$& \cdCommand{stop column density}\\
Grains& No grains& \cdCommand{grains}\\
Spectral resolution& mesh resolution& \cdCommand{set save line width}\\
Background cosmic rays& No& \cdCommand{cosmic rays}\\
Cosmic background& No& \cdCommand{background}\\
\hline
\end{tabular}
\end{table}

The code is also designed to check that its assumptions are not violated.
It should complain if problems occur, if its limits are exceeded, or if
the input parameters are unphysical.  It may print a series of warnings,
cautions, or notes if some limit was exceeded or physical assumption
violated.

\section{Command format}

\subsection{Input and Output}
When executed as a stand-alone program \Cloudy\ reads
\cdFilename{stdin} for input and produces output on \cdFilename{stdout.}
From a command prompt, this
would be done as \cdCommand{cloudy.exe $<$ input $>$ output}
or \cdCommand{cloudy.exe -r prefix} (the latter form will read its
input from \cdFilename{prefix.in} and write its main output to \cdFilename{prefix.out}).
A third option is to supply the \Cloudy\ commands on the command line, e.g.
\cdCommand{cloudy.exe -e test} which will execute the \Cloudy\ command
\cdCommand{test} (the ``smoke test''). This is useful for quickly executing
short scripts. If you want to execute multiple commands, the commands need to be separated
by a semicolon and embedded in single quotes (to protect the semicolon from the shell), e.g.
\cdCommand{cloudy.exe -e 'set continuum resolution 0.1; compile grains'}.
The code is also designed
to be used as a subroutine of other, much larger, programs.
In this case the input stream is entered using
the subroutine calls described in a section of Part 2 of this document.
In either case, this input stream must contain all the commands needed to
drive the program.
The command format is the same whether used as a
stand-alone program or as a subroutine.

\subsection{Command-line format.}
\label{sec:CommandLineFormat}
Commands are entered at the start of a line (i.e. starting in column 1).
They may be abbreviated, as long as the abbreviation is unambiguous,
with a minimum of four characters (unless the whole command is shorter of course).
The command is usually followed by one or more
numbers or keywords.
The script can be either lower or upper case, i.e. it is case insensitive.
All the characters of the command name that are typed will be checked by the parser,
unlike keywords where usually only the first four characters are checked\footnote{This
will be changed in a future version of the code such that also all characters of the keywords will be checked.}.
The code adheres to US spelling where appropriate.

The end of each command line is marked\footnote{Before version 92 a colon (``:'') could also mark an end of line.
This character is needed to specify a path in the Windows environment and
is no longer an end-of-line indicator.} by the end-of-line, or the start of a comment
(see Sect.~\ref{sec:CommentsInInput}).

The command lines can generally be in any order and there is no limit to the length of an input line.
The input stream ends with a blank line
(i.e. an empty line, or a line with a space in the first column), the end-of-file,
or a field of stars (``***'', i.e. a minimum of three consecutive stars starting in the first column).

\subsection{Units}
Most commands use cgs units.
In a few cases common astronomical
nomenclature can be entered (i.e., the luminosity can be specified
as erg s$^{-1}$, in solar units, or even magnitudes).
This syntax varies from command
to command so it is important that the units be checked carefully.

\subsection{Number of commands.}
There is no limit to the number of commands that may be entered.

\subsection{Output as input.}  \Cloudy\ can read its own output
as an input stream.
As described in the section \cdSectionTitle{Output} in
Part 2 of this document, the code
echoes the input command lines as a header
before the calculation begins.
These lines are centered on the page and surrounded by asterisks.

Sometimes a particular model will need to be recomputed.  You can do
this by making a copy of the printed command lines and using this copy as
an input file.  The input parser will handle removal of the leading spaces
and asterisk.  This is mainly a debugging aid.

\subsection{Syntax used in this document}

Sections describing each of the commands are introduced
by examples of their use.

Square brackets indicate optional parameters.
Optional parameters are
shown surrounded by square brackets (``['' and ``]'').
The examples shown
below use the format given in this document.
\begin{verbatim}
# following needs flux density, but frequency is optional
f(nu) = -12.456 [at .1824 Ryd]
#
# the luminosity command has several optional keywords
luminosity 38.3 [solar, range, linear]
#
# the phi(h) command has the range option
phi(h) = 12.867 [range ...]
\end{verbatim}
These square brackets indicate only that the parameters are optional.
The brackets should not be placed on the command line.
They will be totally ignored if they occur.
The above example would actually be entered as follows:
\begin{verbatim}
# following gives flux density at energy of 0.1824 Ryd
f(nu) = -12.456 at .1824 Ryd
#
# the luminosity command with linear keyword
luminosity 2e38 solar linear
#
# the phi(h) command with the range option
phi(h) = 12.867 range 0.1 to 0.2 Ryd
\end{verbatim}

Underscores indicate a space.  Most commands and keywords require four
character matches to be recognized.  Keywords which start with a letter (i.e.\ A$--$Z) must start following a space (or other non-alphabetic character) in order to be recognized.
Only one space is needed between words.

The following is an example with the commands written as they are shown
in this document:
\begin{verbatim}
# blackbody with T=5e4 K, in strict TE
blackbody 5e4 K lte
#
# use ISM radiation field
table ism
\end{verbatim}
The following is how the commands should actually be entered:
\begin{verbatim}
# blackbody with T=5e4 K, in strict TE
blackbody 5e4 K lte
#
# use ISM radiation field
table ism
\end{verbatim}
The space must occur where the underscore is written.

\subsection{and, because nobody ever reads this document\dots.}

The examples of commands that follow show the square brackets
and underscores for
optional parameters and required spaces.
Many people put these characters
into the input stream because they don't read documentation.
As a service to the user, the command-line parser will
usually replace any square brackets
or underscores with the space character when the command lines
are initially read.
The exception is any part of a string that is surrounded by double
quotes.
The string between double quotes is likely to be a file name and
an underscore can occur in such a name.

\subsection{The continue option}

It may not be possible to enter all the required values on a single line
for the \cdCommand{interpolate} and \cdCommand{abundances} commands.  In these two cases the original
command line can be continued on following lines with a series of lines
beginning with the keyword \cdCommand{continue}.  The format on a \cdCommand{continue} line is
unchanged.  There is no limit to the number of \cdCommand{continue} lines that can be
included.   The following
is an example with the abundances command
\begin{verbatim}
abundances he =-1 li=-9 be=-11 b=-9 c=-4.3 n=-5 o=-2.3
continue f=-7 ne =-1.2 na =-3 mg =-8
continue al =-8 si =-8 p=-6 s=-8 cl=-9 ar =-8 k=-6
continue  ca =-8 sc=-9 ti=-7 v=-8 cr=-6.3 mn=-6 fe =-8
continue co =-9 ni =-8cu=-7 zn=-7
\end{verbatim}

\subsection{Numerical input}

Numerical parameters are entered on the command line as free-format
numbers.
Exponential notation can be used.\footnote{Exponential notation could not be used in versions 07 and before.}
Numbers may be preceded
or followed by characters to increase readability.
The strings
``T=1000000K'', ``1000000'', and ``T=1E6'' are equivalent.
A period or full stop (``.'') by itself is interpreted as a character,
not numeral or number.

Default values are often available.
As an example, the \cdCommand{power law} command
has three parameters, the last two being optional.  The following are all
acceptable (but not equivalent) forms of the command;
\begin{verbatim}
power law, slope-1.4, cutoffs at 9 Ryd and 0.01 Ryd
powe -1.0 5
power law, slope=-1.4 .
\end{verbatim}
The last version uses the default cutoffs, i.e., none.  If optional
parameters are omitted they must be omitted from right to left; numbers
must appear in the expected order.

Note that implicit negative signs (for instance, for the slope of the
power law) \emph{do not} occur in any of the following commands.

Table \ref{tab:FreeFormatNumbers} shows how various typed
input numbers will be interpreted.
The first column gives the typed quantity,
the second its interpretation, and the third the explanation.

\begin{table}
\centering
\caption{\label{tab:FreeFormatNumbers}
Interpretation of Numerical Input}
\begin{tabular}{llp{12pc}}
\hline
Typed Quantity& Interpreted as& Why\\
\hline
1e2& 100& Exponentional notation is OK\\
1.2.3& two numbers 1.2 0.3& 1.2 is parsed, then .3\\
.3 3& 0.3 and 3.0\\
10\verb|^|3 & 1000 & \verb|^| acts as exponentiation operator\\
\$temp & variable value & set by \$temp = 1000 previously in input\\
\hline
\end{tabular}
\end{table}

\subsection{Comments}
\label{sec:CommentsInInput}

Comments may be entered among the input data in several ways.
Comments
can be entered at the end of a command line after
a sharp sign (``\#'')
or a double sharp sign (``\#\#'').
Anything on a line after one of these characters is completely ignored.
This can be used to document parameters on a line.
Any line beginning with
a \# or \#\# is also a comment and the entire line is ignored.
Any comment starting with a single sharp sign is ``visible'', which means that
it is echoed in the main output and also included in the file \cdFilename{optimal.in}
(see Sect.~\ref{sec:optimize_file}).
Comments starting with a double sharp sign are ``hidden'' comments and will be omitted in the main output
and the file \cdFilename{optimal.in}.
There is also a \cdCommand{title} command, to enter a title for the simulation.

\subsection{Hidden commands}

A command will be parsed and used by the code but not printed in the
output if the keyword \cdCommand{hide} occurs somewhere on the command line.
This
provides a way to not print extensive sets of commands, like the
\cdCommand{continue}
option on the \cdCommand{continuum} command,
or the \cdCommand{print off} command in an initialization file.

\begin{shaded}
\subsection{\experimental\ Commands for experimental parts of the code}

The code is in continuous development. When new versions are released there
will be new or experimental parts of the code that are still being developed
or which have not been fully debugged. The newly-developed physics is not
included in a calculation which uses the default conditions.
The commands
which exercise these new features are included in this document
and are indicated in two ways.
First by the \experimental\ symbol in the section
header, and second by a grey background. You are welcome to give these
commands a try but should not expect robust results.
\end{shaded}

\subsection{Log vs linear quantities}

Most
quantities are entered as the log of the number but some are linear.
The
following outlines some systematics of how these are entered.
\begin{description}
\item[Temperature.]  \Cloudy\ will interpret a temperature as a log if the number
is less than or equal to 10 and the linear temperature if it is greater
than 10.  Many commands have the optional keyword \cdCommand{linear} to force numbers
below 10 to be interpreted as the linear temperature rather than the log.

\item[Other parameters.]  The pattern for other quantities is not as clear as
for the case of temperature.  Often quantities are interpreted as logs if
negative, but may be linear or logs if positive (depending on the command).
Many commands have the keywords \cdCommand{log} and \cdCommand{linear} to force one or the other
interpretation to be used.

\end{description}

Using the new notation \verb|10^3.5| as a linear number is equivalent
to the form \cdCommand{3.5 log}, and will be required to replace the
\cdCommand{log} keyword in future versions of \Cloudy.

\begin{shaded}
\subsection{\experimental\ The \cdCommand{help} command}

This command prints helpful information about Cloudy and exits.  Don't
expect much as yet, beyond a recommendation to read this document!

\end{shaded}

\subsection{An example}

Specific commands to set boundary conditions for a simulation are
discussed in the following sections.
As a minimum, the hydrogen density,
the shape and intensity of the incident radiation field,
and possibly the starting radius, must be specified to compute a model.
As an example, a simple model
of a planetary nebula could be computed by entering
the following input stream.
\begin{verbatim}
title "this is the input stream for a planetary nebula"
#
# set the temperature of the central star
blackbody, temperature = 1e5 K
#
# set the total luminosity of the central star
luminosity total 38 # log(Ltot)- ergs/s
radius 17   # log of starting radius in cm
hden 4      # log of hydrogen density - cm^-3
sphere # this is a sphere with large covering factor
\end{verbatim}

\section{Reading and Writing Files}
\label{sec:ReadingWritingFiles}

When \Cloudy\ is running, it needs to read many files and will usually also
write new files to disk. In this section we will describe the rules used by
the code to decide the location of these files. The skinny is that input files
are always searched along a search path and output files are written in the
local directory. Only one exception exists: the main input script, which must
be located in the local directory. The default search path is to look first in
the local directory, then the data directory of the \Cloudy\ distribution.
Below we will give more details.

The first thing the code needs is the main input script. The name of this file
can be supplied to the code on the command line (e.g., using the
\cdVariable{-r} flag). This file always needs to be in the local directory
(a.k.a.\ current working directory). The code will not work correctly if you
try to supply a filename in a subdirectory or using an absolute path, and will
abort in that case. All output files written by the code will be in the local
directory as well. This of course includes the main output file.

It is sometimes necessary to read or write external files whose names
are specified on a command line.  File names are entered inside pairs of
double quotes, as in \cdFilename{"name.txt"}.

The command parser first checks whether a quote occurs anywhere on the
command line.  If one does occur then the parser will search for a second
pair of quotes and use whatever text lies between as a filename.  The code
will stop with an error condition if the second of the pair of quotes is
not found or if the file cannot be opened for reading or writing.

If an input file is supplied in this way, the rule is to always search for the
file along a search path. This allows the user more freedom to organize the
files. The default search path is to look in the local directory first, and if
it is not found there, look in the \Cloudy\ data directory. The code will
issue a caution if multiple matches with the same file name were found along
the search path. It will then continue using the first match. An example is
shown below:
\begin{verbatim}
CAUTION: multiple matches for file continuum_mesh.ini found:
   ==/path/to/run/continuum_mesh.ini==
   ==/path/to/cxx.yy/data/continuum_mesh.ini==
Using the first match.
\end{verbatim}
The user can override or extend the default search path. This is described in
more detail below in Section~\ref{sec:SearchPath}.

As stated before, \Cloudy\ will always write output files in the local
directory. If you try to supply a different path between double quotes (i.e.,
a file in a subdirectory or an absolute path) the code will abort, as in the
following example:
\begin{verbatim}
save continuum "subdir/name.com" # ERROR, this will abort!
\end{verbatim}

The search path described above will also be used when \Cloudy\ is reading
data files from its distribution. This allows the user to place modified data
files in the local directory and leave the files in the main data directory
untouched. This is useful if you want to limit the scope of the modification
to the current run only (to be precise, all subsequent runs in the directory
that contains the modified file). One example would be when you want to alter
the resolution of the \Cloudy\ frequency mesh by modifying the file
\cdFilename{continuum\_mesh.ini} (see
Section~\ref{Hazy2-sec:ChangingMeshResolution} in Part 2 of \Hazy). If you
place the modified file in the local directory, it will only affect runs in
that directory and not globally.

\section{The Search Path for Input Files}
\label{sec:SearchPath}

The code uses a search path to locate input and data files. 
The default search path is to look in the local directory
first and then in the data directory included in the distribution. 
The code creates the default search path on the fly when the code starts up. 
The makefile sets the path to the root of the Cloudy installation. 
This section described how to set your own search path if you want to extend
our default setup..

The user
can override that setting by defining the environment variable
\cdVariable{CLOUDY\_DATA\_PATH}. How you can do that depends on the operating
system and the shell you are using. For csh/tcsh under Linux/UNIX use:
\begin{verbatim}
setenv CLOUDY_DATA_PATH ".:/path/to/cloudy/data" # (double quotes are optional)
\end{verbatim}
For bash/sh/ksh/etc... you should use:
\begin{verbatim}
export CLOUDY_DATA_PATH=".:/path/to/cloudy/data" # (no spaces around the '='!)
\end{verbatim}
These methods should also work under Mac OS X. The environment variable will
be read when the code starts up and then override the standard setting
compiled into the code. If you subsequently undefine the environment variable,
the code will revert back to using the standard search path.

The environment value \cdVariable{CLOUDY\_DATA\_PATH} supports a search path
with multiple components, similar to the search paths used in UNIX. You can
supply multiple directory paths, separated by a colon (':') on UNIX and Mac
systems, or separated by a semicolon (';') on Windows systems. When Cloudy
searches for a file, it will search in each of the component directories until
it finds the file. The search will be done in the same order as they were
defined in the environment variable. The code will {\em not} do a recursive
descent into subdirectories of the search path components. The examples shown
above would be equivalent to the default setting as the special path
\cdFilename{``.''} is shorthand for the local directory, irrespective of where
you are.

When you modify the search path, you should realize that the default data
directory must always be part of the search path. The code will not function
otherwise! It is also strongly recommended to keep the local directory as part
of the search path. To simplify this, \Cloudy\ supports the special path
component \cdFilename{``+''} which will be expanded to the default search path
for that release.

A typical example would be this:
\begin{verbatim}
export CLOUDY_DATA_PATH="+:/path/to/local_data"
\end{verbatim}
This search path would consist of the default search path (i.e., the local
directory and the \Cloudy\ data directory) followed by a custom directory
\cdFilename{/path/to/local\_data} where you can store local files, such as
stellar atmosphere grids, or custom grain opacity files. This avoids
``polluting'' the standard data directory and also makes it easier to port the
files from one release to the next. You can make multiple local data
directories and combine them in custom sequences. One example would be one
directory containing general local files, while the other contains files
specifically needed for a non-standard spectral resolution run (such as binary
stellar atmosphere grids, or grain opacity files). Such an example could look
like this:
\begin{verbatim}
export CLOUDY_DATA_PATH="/path/to/non_standard_data:+:/path/to/local_data"
\end{verbatim}
Note that we put \cdFilename{/path/to/non\_standard\_data} up front (this is
the directory containing data for a non-standard spectral resolution). This
assures that the non-standard resolution version of a file is picked up first,
before the standard version.

You can verify your search path using the \cdCommand{print path} command
described in Section~\ref{sec:PrintPath}.

\section{The \cdCommand{init} command}

\noindent
This is a special command that tells the code to read a set of commands
stored in an ancillary file.  This allows frequently-used commands to be
stored in a separate file.
The \cdCommand{init} command is fully described in later sections below.

The filename can be specified within a pair of double quotes, as in
\cdFilename{"ism.ini"}.
The default name for an initialization file is
\cdFilename{cloudy.ini}.
There is no limit to the number of commands that can be in an
initialization file.

This provides an easy way to change the default behavior of the code.
For instance, many of the elements now included in \Cloudy\ have
negligible abundances and the code will run a bit faster
if they are turned off with
the \cdCommand{element off} command.
Also, only about half of these
elements were included before version 86 of the code.
The file \cdFilename{c84.ini} in the \Cloudy\ data directory
which will turn off many of these elements.
The \cdFilename{c84.ini} file contains
the following commands:
\begin{verbatim}
print off hide
element Lithium off
element Beryllium off
element Boron off
element Fluorine off
element Phosphor off
element Chlorine off
element Potassium off
element Scandium off
element Titanium off
element Vanadium off
element Chromium off
element Manganese off
element Cobalt off
element Copper off
element Zinc off
print on
\end{verbatim}

The current version of the code would only include those elements present
in version 84 if the command
\begin{verbatim}
init "c84.ini"
\end{verbatim}
were entered in the input stream.

A series of \cdFilename{*.ini} files are included in the
data directory included in the \Cloudy\ distribution.
Do an \cdFilename{ls *.ini} within the data directory
to list the available files. Note that not all these files
are init files, e.g., \cdFilename{continuum\_mesh.ini} is not
an init file in the sense we discussed here.
Comments at the start of the files describe
their purpose.

\section{Random numbers}

Some commands in \Cloudy\ need random numbers to work. Like all computer
codes, \Cloudy\ uses a pseudo random number generator (PRNG) to create these
numbers. A PRNG creates a deterministic sequence of numbers starting from a
given seed. However, on each invocation the bits in the number are so
efficiently scrambled that for all intents and purposes they can be considered
a truly random sequence of numbers. The PRNG in \Cloudy\ is by default a fully
vectorized version of xoshiro256** (Blackman and Vigna, 2018, in
press)\footnote{http://xoshiro.di.unimi.it.}, while the random numbers with a
normal distribution are now generated using the Ziggurat algorithm
\citep{MarsagliaTsang00}. Both methods are very fast. An additional advantage
is that the new code is fully aware of parallelization in the code, meaning
that parallel ranks created with MPI or fork will automatically have a
completely different sequence of random numbers. \Cloudy\ generates a random
seed at the start of execution by default (when available derived from
\cdFilename{/dev/urandom}, otherwise using the system time). The executable
accepts the {\tt -s~[~seed~]} flag, which will set a fixed seed for the random
number generator and is intended for debugging and testing purposes. The
parameter is a 64-bit seed in hexadecimal form. If the seed is omitted, a
default fixed value will be used. It is not recommended to use this flag in
normal \Cloudy\ runs.
