\chapter{THE GRID COMMAND}
% !TEX root = hazy1.tex
\label{sec:CommandGrid}

\section{Introduction to grids}

The \cdCommand{grid} command varies one or more of the input parameters
to compute a grid of models.
It is actually a form of the \cdCommand{optimize} command and uses much
of the same code.
It was added by Ryan Porter and its first use was
described in \citet{Porter2006}.
The grid models will be run in parallel on multi-core machines.

In a typical setup one or more parameters will be varied and output saved.
A typical \cdCommand{grid} run might look something like the following:
\begin{verbatim}
# set up a simple AGN calculation
table agn
# vary the ionization parameter
ionization parameter -1 vary
# this grid command says how to vary the value of the ionization parameter in the
# preceding command.  Go from log U =-2 to log U = 0 in 0.5 dex steps
grid -2 0 0.5
stop column density 19
hden 10
iterate to convergence
# save the grid points, the value of the ionization parameter of each model
save grid "gridrun.grd" last no hash
# save intensities of selected lines
save line list "gridrun.lin" "LineList_BLR.dat" last no hash
\end{verbatim}

This script computes a series of AGN models in which the ionization parameter
is varied between two limits.
The \cdCommand{save grid} command saves the values of the ionization parameter
and gives some information about the calculation.
The \cdCommand{save line list} command saves a set of lines for each grid point.

\section{Running the code with grids}

\subsection{Command line setup}
It is {\em mandatory} to run \Cloudy\ using one of the
following commands to use the \cdCommand{grid} command:
%
\begin{verbatim}
# non-MPI runs, using fork
cloudy.exe -r gridrun
# alternative
cloudy.exe -p gridrun

# MPI runs
mpirun -np <n> /path/to/cloudy.exe -r gridrun
# alternative
mpirun -np <n> /path/to/cloudy.exe -p gridrun
\end{verbatim}
This assumes that the input script is in the file \cdFilename{gridrun.in}. 

Using the \cdCommand{-r} or \cdCommand{-p} command line switches is
necessary because the code needs to know the name of the input script in order
to generate intermediate files with its name embedded. It is also necessary
for the correct functioning of the code in MPI mode.
The \cdCommand{-r} option is included in the recommended setup for
running \Cloudy.

\subsection{Grids on parallel machines}

On most systems the \cdCommand{grid} command will run in parallel by default using all
available cores / hyperthreads. This includes all UNIX systems, as
well as Cygwin under Windows.
On most machines there are twice as many threads as CPUs, and the
additional threads do not add much computing power.
Tests show that a Mac will become unresponsive if all threads are used, so for OS X
we use all cores rather than all threads, by default.

\emph{Grids on MPI machines}
By default the \cdCommand{grid} command will use the UNIX fork function to 
run multiple jobs, and nothing more need be done.
MPI can be used to run grids on distributed systems.
When the code is
compiled with MPI support enabled (see a section of \Hazy~2),
the \cdCommand{grid} command will run
in parallel mode on any (distributed) MPI cluster. This is the preferred method if you
want to run very large grids and you need more cores than a single node can provide.
The second example above shows how to run on a typical MPI-capable machine.

\subsection{Forcing sequential runs}
You can disable the parallel behavior by adding
the keyword \cdCommand{sequential} to the \cdCommand{grid} command. 
The \cdCommand{sequential} option causes the grid to be executed using only a
single core rather than in parallel mode. This option is ignored when running
in MPI mode.

\subsection{Specifying the number of CPUs}
You can
choose a different number of cores by adding the keyword \cdCommand{ncpus}
at the end of the \cdCommand{grid} command. 
The \cdCommand{ncpus} option allows you to set the number of cores that will
be used in parallel grid runs. This option is only effective on systems that
support parallelization based on the \cdRoutine{fork()} system call. It is
ignored in MPI runs since the number of cores is set as an option to
\cdCommand{mpirun} in that case. 
The number of CPUs should be added as a fourth number on
the command line, e.g. as follows:
\begin{verbatim}
blackbody 5.e4 vary
grid 4 5 0.1 ncpus 4  ## limit # of cpus to 4
\end{verbatim}

\section{Grid start point, end point, increment [ linear ]}

Parameters for those commands with the \cdCommand{vary} keyword
(see Table \ref{tab:CommandWithVaryOption}
on page \pageref{tab:CommandWithVaryOption}) can be varied
within a grid.
Each
command with a \cdCommand{vary} option must be followed by a \cdCommand{grid} command.
The value
entered on the command with the \cdCommand{vary} option is
ignored but must be given
to satisfy the command parser.

The \cdCommand{grid} command line must have three numbers.
The first and second
are the start and end points of the range of variation.
The last number
is the value of the increment for each step.
The increment can be either positive or negative.
This behaves much the same
way as a \emph{do} loop in Fortran or a \emph{for} loop in C.
The \cdCommand{grid} command always changes variables from the start point
to the end point.
In other words, the start point must be less than the end point when
the increment is positive, while, when the increment is negative, the start
point must be larger than the end point. Note that the
grid points are calculated in random order, so it no longer
matters if you use a positive or negative increment\footnote{In versions
C10 and earlier of the code, the grid points were calculated in sequential
order (at least for non-MPI runs) so it made a difference then. The option of
choosing a negative increment is maintained for backward compatibility.}.
The following varies the
blackbody temperature over a range of $10^4$ to $10^6$~K with three points.
\begin{verbatim}
# the blackbody command with vary option, the 1e5 K is needed to
# get past the command parser, but is otherwise ignored
blackbody 1e5 K vary
# this varies the blackbody temperature from 1e6 to 1e4 K with -1 dex
# steps, so that 1e6K, 1e5K, and 1e4K will be computed
grid, range from 6 to 4 with -1 dex increments
\end{verbatim}
The following is a 2D grid in which both the ionization parameter and
metallicity are varied.
There will be five values of the ionization
parameter and three values of the metallicity computed.
\begin{verbatim}
ionization parameter -2 vary
grid range from 0 to -4 with -1 dex steps
metals 0 vary
grid range from -1 to 1 with 1 dex steps
\end{verbatim}

For nearly all commands, the quantity will be varied logarithmically (current
exceptions are the \cdCommand{illumination}, \cdCommand{ratio alphox},
\cdCommand{dlaw}, and \cdCommand{fudge} commands). If the quantity is varied
logarithmically, the lower / upper limit and the step size also need to be
given as logarithms, as shown above. If the keyword \cdCommand{linear} is
included on the \cdCommand{grid} command, then these numbers will be
interpreted as linear quantities. As an example, the following will produce a
grid of models with a constant electron temperature of 5000, 10000, 15000, and
20000~K.
\begin{verbatim}
constant temperature 4 vary
grid range from 5000 to 20000 step 5000 linear
\end{verbatim}

For some commands the unit of the quantity will be implicitly changed by the
grid code. An example is the \cdCommand{radius} command. You can enter a
radius in parsec using the \cdCommand{parsec} keyword. However, the grid code
will always vary the radius in cm. In this case the lower / upper limit will
also need to be the (logarithm of) a radius in cm. There is currently no way
to overrule this. In general the lower / upper limit needs to have the exact
same units as are used on the command line that is generated by the grid,
\emph{not} the command line as originally typed by the user.

\section{Grid list [ linear ] ``filename''}

This \cdCommand{list} keyword allows parameter values to be individually
listed in an external file that must be given in quotes on the same line. The
file should not contain any other text, but may contain comments starting with ``\#''. The numbers should
be separated by whitespace and/or newlines. The numbers will be interpreted as
log values unless the \cdCommand{linear} keyword is included. This option is
useful when a regularly-spaced grid as outlined above is too restrictive.

\section{Beware the grid command treatment of temperatures!!}
\label{sec:GridTemperatureGotcha}
%:GridTemperatureGotcha
The following will crash with an fpe
\begin{verbatim}
constant temperature 4 vary
grid range from 5000 to 20000 step 5000  ## wrong, this will crash!
\end{verbatim}
This is because of the rule stated above that the \cdCommand{grid} command
treats temperature ranges as logs unless the keyword
\cdCommand{linear} occurs.  

\section{Grid output options}
\label{sec:GridOutputOptions}

Several \cdCommand{save} output
options were developed to
retrieve information from \cdCommand{grid} runs.
The \cdCommand{save XSPEC} command
will generate FITS files for input into XSPEC (\citealp{Porter2006}).
The
\cdCommand{save grid} and \cdCommand{save line list} commands
save the grid points and emission-line intensities.

When computing a grid, the default is for save files to not overwrite
results of previous grid points (to not ``clobber'' themselves) and
to append the results to the save file instead.
Conversely, in optimizer runs the default
is for save files to overwrite one another since you
are usually not interested in output along the unpredictable convergence path.
That way you only get the save output from the fully converged model.

When you include the \cdCommand{separate} keyword on a \cdCommand{save}
command, the output of each grid point will be written to a separate file. See
Section~\ref{sec:save:separate} for further details.

In the output from \cdCommand{save} commands the default behavior is to put
separators consisting of hash marks in between the output from different
iterations (unless the \cdCommand{no hash} option is used). In grid runs this
behavior is slightly changed. There will be an additional separator at the
very end of the output, which will also have the additional text
``GRID\_DELIMIT'' appended, as well as the number of the grid point in the
format ``grid000000000''. This makes it easy to find back the output from a
specific grid point, as well as identify where there is output missing in case
some of the grid points have serious problems.

During a grid run the main output for each grid point will be written to a
separate file with names like \cdFilename{grid000000000\_yoursimname.out},
\cdFilename{grid000000001\_yoursimname.out}, etc. (assuming your input script
is named \cdFilename{yoursimname.in}). The meaning of the grid index embedded
in the filename can be found with the \cdCommand{save grid} command described
in Section~\ref{sec:save:grid}. At the end of the grid run these files will be
gathered into a big main output file called \cdFilename{yoursimname.out} and
the intermediate files will be deleted. For huge grids this operation can take
a very long time. However, if you include the \cdCommand{separate} option on
any of the \cdCommand{grid} commands, the gathering step will be skipped and
the main output will remain in separate files, thus speeding up the
calculation considerably.

\section{Other grid options}
\label{grid:other:options}

The \cdCommand{no vary} command can be used to ignore all
occurrences of the \cdCommand{grid} command and \cdCommand{vary}
option in an input stream.

The \cdCommand{repeat} option on the \cdCommand{grid} command
causes the expected number of grid
steps to be computed but the initial value of the variable is not
incremented.
This is mainly a debugging aid. 

The \cdCommand{cycle} option causes the grid to be executed multiple times. If
no additional number is found on the command line, the grid will be executed
twice. You can also add a fourth number on the command line, which will be the
number of times the grid will be executed. That number needs to be at least
two. This is a debugging aid and should never be used for production work.

\section{Notes on various commands}

\subsection{Constant temperature}
Section \ref{sec:GridTemperatureGotcha}
describes the rules governing how the temperature can
appear on the \cdCommand{constant temperature} command.
