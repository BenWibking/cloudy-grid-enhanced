\chapter{INCIDENT RADIATION FIELD LUMINOSITY}
\label{sec:IncidentRadiationFieldLuminosity}
% !TEX root = hazy1.tex
\section{Overview}

All commands setting the intensity or luminosity of the
incident radiation field are
defined in this Chapter.

\section{Intensity and luminosity cases}
The brightness of the incident radiation field can be specified
as either an intensity, the energy per unit area of cloud,
or as a luminosity, the power emitted by the central
source of radiation into $4 \pi \sr$.
The \cdTerm{intensity case} and \cdTerm{luminosity case}
are described on page \pageref{sec:IntensityLuminosityCases}.
Each of the following commands is listed as an intensity or
luminosity command.
This distinction is important because the inner radius of the
cloud must also be specified in the luminosity case.

\section{The range option}

Many of the intensity/luminosity commands specify the number of photons
or integrated energy \emph{in hydrogen-ionizing radiation}
$(1 \Ryd \le h\nu \le$ \egamry ).
Other
energy intervals can be specified with the \cdCommand{range} option, an optional keyword
that can appear on most intensity / luminosity commands.

When the keyword \cdCommand{range} appears there are an
additional two parameters,
the low- and high-energy limits to the energy range in Rydbergs.
These
appear as the second and third numbers on the line
(the first number gives
the intensity or luminosity).
The position of the keyword \cdCommand{range} on the
command line does not matter but the order of the numbers on the line does
matter.
If both range parameters are omitted then the low
(\emm ) and high (\egamry )
energy limit of the incident radiation field will be
substituted.  If both energies are specified then the
second number must be larger than the first.
If only
one parameter appears then only the lower limit of the range
will be changed
and the high-energy limit will be left at its default of
\egamry .
If the first optional number is negative or
the keyword \cdCommand{log} appears then
\emph{both} of the extra numbers are interpreted as logs.
If you want to set the lower limit of the range to the low
energy limit \emm\ of the incident radiation field, but want
to supply your own upper limit, you can enter 0 followed by
the upper limit if you use linear numbers, or $-10$ followed by
the upper limit if you use log numbers ($-10$ is actually any
number $\leq$ log(\emm)).

If \cdCommand{range total}, or simply \cdCommand{range} or \cdCommand{total},
appears with no parameters then the
full energy range considered by the program,
\emm\ to
\egamry , will be used.
In this case the number is the total integrated (bolometric) intensity
or luminosity.

The following shows examples of the range option for the
\cdCommand{luminosity} command.
By default the \cdCommand{luminosity} command has a single parameter,
the log of the luminosity [erg s$^{-1}$] in hydrogen-ionizing
radiation ($1 \Ryd \le  h\nu <$ \egamry ).
The ``;'' symbol is used to terminate the line in one
case.
\begin{verbatim}
# this will use the default range, only ionizing radiation
luminosity 38 #the log of the luminosity in erg s^-1

# either will be the total luminosity
luminosity total 38
luminosity 33.4 range
luminosity 33.4 range total

# this will be the luminosity in visible light
luminosity 37.8 range .15 to .23 Ryd

# the luminosity in radiation more energetic than 0.1 Ryd
luminosity 38.1 range -1

# this will be the luminosity in non-ionizing radiation
luminosity 39.8 range 0 1
\end{verbatim}

\section{Absolute [visual, bolometric] magnitude -2.3}

It is possible to specify the integrated or monochromatic luminosity
in ``magnitudes,'' a quaint unit of historical interest.
One of the keywords
\cdCommand{bolometric} or \cdCommand{visual} must also appear.
The absolute bolometric magnitude
$M_{bol}$ is related to the total luminosity by (\citealp{Allen1976}, page 197)
\begin{equation}
L_{total}  = 3.826 \times 10^{33}  \times 10^{\left( {4.75 - M_{bol} }
\right)/2.5}\,
 [\ergps].
\end{equation}
The absolute visual magnitude $M_V$ is approximately related to the
monochromatic luminosity per octave at 5550 \AA\
by (\citealp{Allen1976}, page 197)
\begin{equation}
\nu \,L_\nu  \left( 5500\mathrm{\AA}\right) \approx 2.44 \times 10^{35}
\times 10^{ - M_V /2.5}\,
[\ergps].
\end{equation}
The conversion between monochromatic luminosity per octave $\nu L_{\nu}$ and absolute
visual magnitude $M_V$ is approximate,
with typical errors of roughly a percent.
This is because \Cloudy\ assumes that the V filter has an
isophotal wavelength of 5550\AA\ and does not actually integrate
over the incident radiation field using
a $V$-filter transmission function.

This is a luminosity command.

\section{Energy density 5e4 K [linear]}

This specifies the energy density [K] of the incident radiation field.
The number is the equivalent energy-density temperature, defined as
$T_u  = \left( {u/a} \right)^{1/4} \K$
where $u$ is the total energy density in all radiation
[erg cm$^{-3}$] and $a$ is the Stefan radiation-density constant.
The number is interpreted as the log of the temperature if it is less than or equal to 10
or the keyword \cdCommand{log} is present,
and as a linear number otherwise.
The optional keyword \cdCommand{linear} forces the
number to always be interpreted as a linear temperature.

This is an intensity command.

\section{f(nu) = -12.456 [at .1824 Ryd]}

This specifies the monochromatic intensity at an arbitrary energy.  The
first number is the log of the monochromatic mean intensity at the
illuminated face of the cloud,
$4\pi J_\nu$ (with units erg
s$^{-1}\mathrm{Hz}^{-1} \mathrm{cm}^{-2})$, where
$J_{\nu}$ is the mean intensity of the incident radiation field
per unit solid angle.

The optional second number is the frequency in Rydbergs where
$4\pi J_\nu$ is specified.
The default is 1 Ryd.
In the example above the incident radiation field is
specified at 0.1824 Ryd = 5000\AA.
The frequency can be any within the energy
band considered by the code,
presently \emm\ to \egamry .
If the energy is less than or equal to zero then it is interpreted as the
log of the energy in Rydbergs, and as the linear energy itself if positive.

This is an intensity command.

\section{Intensity 8.3 [range, linear]}
\label{sec:IntensityCommand}

This specifies the integrated mean intensity,
$4\pi J$, [erg cm$^{-2}\mathrm{s}^{-1}$] at the
illuminated face of the cloud
\begin{equation}
4\pi J= \int_{v1}^{v2}4\pi J_v \; dv \,[\ergpscmps] .
\end{equation}
This is the per unit area equivalent of the \cdCommand{luminosity}
command.
The number is the log of the intensity unless the optional
keyword \cdCommand{linear} appears.
Unlike the majority of the commands, the first
five characters of the line must be entered.

The default range is over hydrogen-ionizing energies
(1 Ryd  $\le hv\le$ \egamry ).
The range option can be used to adjust
the values of $v_1$ and $v_2$.

Some of the interstellar medium and photo-dissociation region (PDR)
literature specifies the incident radiation field in units of the
\citet{Habing1968} field (see, for instance, \citealp{Tielens1985a},
\citealp{Tielens1985b}).
This
radiation field has an integrated intensity of
$1.6\times 10^{-3} \ergpscmps$ between
the limits of 6 and 13.6 eV (\citealp{Tielens1985a}; \citealp{Hollenbach1991}).
This integrated intensity is sometimes referred
to as $Go$.
The incident radiation field described by Tielens and Hollenbach,
but with an
intensity of 1 $Go$, could be generated with the commands:
\begin{verbatim}
# a B star with an intensity roughly the Habing 1968 radiation field
blackbody 3e4 K
intensity -2.8, range 0.44 to 1 Ryd
# remove all H-ionizing radiation
extinguish by 24, leakage = 0
\end{verbatim}
This set of commands sets the shape of the Balmer continuum to
that of a hot blackbody, sets the intensity to the Habing value,
and then extinguishes
all hydrogen-ionizing radiation,
as is assumed in the PDR literature.

This is an intensity command.

\section{ionization parameter $= -1.984$}

The ionization parameter is the dimensionless ratio of hydrogen-ionizing
photon to total-hydrogen densities.  It is defined as
\begin{equation}
U \equiv \frac{{Q\left( {\mathrm{H}} \right)}}{{4{\kern 1pt} \pi \,r_{\mathrm{o}}^2
\,n\left( {\mathrm{H}} \right)\,c}} \equiv \frac{{\Phi \left( {\mathrm{H}}
\right)}}{{n\left( {\mathrm{H}} \right)\,c}}%(6)
\end{equation}
(AGN3, equation 14.7, page 357).
Here $r_o$ is the separation [cm] between
the center of the source of ionizing radiation and the illuminated face
of the cloud, $n$(H) [cm$^{-3}$] is the
total\footnote{Before version 65 of the code the
electron density was used rather
than the hydrogen density.
Before version 75 $n$(H) was the atomic/ionic
hydrogen density, and did not include molecules.}
hydrogen density (ionized, neutral,
and molecular), $c$ is the speed of light,
$Q(\mathrm{H})$ [s$^{-1}$] is the number of
hydrogen-ionizing photons emitted by the central object,
and $\Phi(\mathrm{H})$
[cm$^{-2}\, \mathrm{s}^{-1}$] is the surface flux of ionizing photons.
The number is interpreted
as the log of $U$ unless the keyword \cdCommand{linear} appears.
The ionization parameter
is a useful quantity in plane-parallel, low-density, constant-density,
models, because of homology relations between models with different photon
and gas densities but the same ionization parameter (see \citealp{Davidson1977}).

This is an intensity command.

\section{L(nu) = 24.456 [at .1824 Ryd]}

This sets the monochromatic luminosity
$L_{\nu}$ [erg s$^{-1} \mathrm{Hz}^{-1}$] of the central object.
The first number is the log of the luminosity.
The optional
second number is the frequency in Rydbergs where $L_{\nu}$ is specified.  The default is 1 Ryd.
In the example above the incident radiation field is specified at
$0.1824 \Ryd = 5000 $\AA.
The frequency can be any within the energy band considered
by the code, presently \emm\ to \egamry .  If the energy is
less than or equal to zero then it is interpreted as the log of the energy
in Rydbergs, and the linear energy itself if positive.

This is a luminosity command.
\label{sec:LuminosityCommand}

\section{luminosity 38.3 [solar, total, range, linear]}

The number is the log of the integrated luminosity\footnote{Before version 83 of the code, the luminosity command was used to
enter both luminosity and intensity.  The code decided between the two by
checking on the resulting ionization parameter.  There are now separate
intensity and luminosity commands.} emitted by the central
object into $4\pi$~sr,
\begin{equation}
L = 4 \pi R_{star}^2 \int_{\nu _1 }^{\nu _2 } {\pi F_\nu
\, d\nu } [\ergps ].% (7)
\end{equation}
The default range is over hydrogen-ionizing energies ($1\mathrm{ Ryd} \le
h\nu\le$ \egamry ).  The \cdCommand{range} option can be used to adjust
the values of $\nu_1$ and $\nu_2$.

If the optional keyword \cdCommand{solar} appears
the number is the log of the \emph{total} luminosity relative to the sun. 
It is the log of the total luminosity if the keyword \cdCommand{total} appears.
If the
\cdCommand{linear} keyword is also used then the quantity will be the luminosity itself
and not the log.
The \cdCommand{range} option cannot be used if the luminosity is the
total luminosity or it is specified in solar units.
It will be ignored
if it appears.

The following are examples of the luminosity command.
\begin{verbatim}
# log of luminosity (erg/s) in ionizing radiation
luminosity 36

# roughly the Eddington limit for one solar mass
luminosity total 38

# both are a total luminosity 1000 times solar since
# solar specifies the total luminosity relative to the sun
luminosity solar 3
luminosity linear solar 1000

# this will be the luminosity in visible light
luminosity 37.8 range .15 to .23 Ryd
\end{verbatim}
This is a luminosity command.

\section{nuF(nu) = 13.456 [at .1824 Ryd]}

This command specifies the log of the monochromatic mean intensity
per octave $4\pi \nu J_\nu  $ [erg s$^{-1}$ cm$^{-2}$]
at the illuminated face of the cloud.
Here $J_{\nu}$ is the
mean intensity of the incident radiation field.

The optional second number is the energy (Ryd) where
$4\pi \,\nu J_\nu  $ is specified.
The default is 1 Ryd.  In the example above the incident radiation field
is specified at $0.1824 \Ryd = 5000\AA$.
The energy can be any within the
energy band considered by the code,
presently \emm\ to \egamry .
If the energy is less than or equal to zero it is interpreted as the
log of the energy in Rydbergs.  It is the linear energy if it is positive.

This is an intensity command.

\section{nuL(nu) = 43.456 [at .1824 Ryd]}

This command specifies the monochromatic luminosity per octave
$\nu L_\nu$ [erg s$^{-1}$].
The first number is the log of the luminosity radiated by the central
object into $4\pi$~sr.  It can be expressed at an arbitrary photon
energy but the default is 1 Ryd.

The optional second number is the energy (Ryd) where $L_\nu$ is
specified.
In the example above the incident radiation field is specified
at 0.1824 Ryd = 5000\AA.
The frequency can be any within the energy band considered by
the code, presently \emm\ to \egamry .  If the energy is less
than or equal to zero, it is interpreted as the log of the energy in
Rydbergs, and the linear energy if positive.

This is a luminosity command.

\section{phi(H) = 12.867 [range\dots]}

This command specifies the log of $\Phi(\mathrm{H})$,
the surface flux of
hydrogen-ionizing photons [cm$^{-2}\ \mathrm{s}^{-1}$]
striking the illuminated face of the cloud.
It is defined as
\begin{equation}
\Phi \left( {\mathrm{H}} \right) \equiv \frac{{Q\left( {\mathrm{H}} \right)}}{{4\,\pi
\;r_{\mathrm{o}}^{\mathrm{2}} }} \equiv \frac{{R_{star}^2 }}{{r_{\mathrm{o}}^2
}}\;\int_{\nu _1 }^{\nu _2 } {\frac{{\pi \,F_\nu  }}{{h\,\nu }}\;d\nu }
 [\mathrm{cm}^{-2}\ \mathrm{s}{^{-1}]}
\end{equation}
and is proportional to the optical depth in excited lines,
such as the Balmer lines (\citealp{FerlandNetzerShields1979}; AGN3).
The range option can be
used to change the default energy range,
given by the values of $\nu_{1}$ and $\nu_{2}$.

This is an intensity command.

\section{Q(H) = 56.789 [range\dots]}

This is the log of the total number of ionizing photons emitted by the
central object [s$^{-1}$]
\begin{equation}
Q\left( {\mathrm{H}} \right) = 4\,\pi \;R_{star}^2 \;\int_{\nu _1 }^{\nu _2
} {\frac{{\pi \,F_\nu  }}{{h\,\nu }}\;d{\kern 1pt} \nu } . %(9)
\end{equation}
The default energy range is 1 Ryd to \egamry\
and the \cdCommand{range} option
can be used to change the energy bounds $\nu_1$ and $\nu_2$.
The photon flux (the
number of photons per unit area of cloud surface) can be specified with
the \cdCommand{phi(H)} command\footnote{Before version 83
of the code the \cdCommand{phi(H)} and \cdCommand{Q(H)} commands were the
same.
The code decided which was specified by checking the order of
magnitude of the resulting ionization parameter.
These are now two different commands.}.

This is a luminosity command.

\section{ratio -3.4 0.3645 Ryd to 147 Ryd [alphaox, log]}

This specifies the intensity of a second radiation field
(referred to as the current radiation field) relative to the
intensity of the previous
radiation field.
The ratio of the intensities
$4\pi J_\nu$ (erg cm$^{-2}\ \mathrm{s}^{-1}\ \mathrm{Hz}^{-1}$)
of the current to the previous radiation field is given
by the first number on the command line.
It is assumed to be the linear ratio unless it is
less than or equal to zero, in which case it is interpreted as a log.
If the keyword \cdCommand{log} appears then the
positive number is interpreted
as the log of the ratio.

The second parameter is the energy in Rydbergs where the previous
radiation field is evaluated and the optional third parameter
is the energy
where the current radiation field is evaluated.
If the second energy is not
entered then the same energy is used for both.
The following is an example
of using the \cdCommand{ratio} command to simulate the
spectral energy distribution of a typical quasar.
\begin{verbatim}
blackbody 5e4 K        # the big blue bump
ionization parameter -2 # its ionization parameter
table power law         #an alpha =-1 power law
# now set intensity of power law relative to bump at 1 Ryd
ratio 0.001 at 1 Ryd
\end{verbatim}
This command was introduced to provide a mechanism to specify the optical
to X-ray spectral index $\alpha_{ox}$.
This is defined here as in \citet{Zamorani1981}, except for a difference in sign convention.
Here $\alpha_{ox}$ is the spectral
index which would describe the continuum between 2 keV (147 Ryd) and
2500\AA\ (0.3645 Ryd) if the continuum could be described
as a single power-law, that~is,
\begin{equation}
\frac{{f_\nu  \left( {2\;{\mathrm{keV}}} \right)}}{{f_\nu  \left(
{2500\;{\mathrm{{\AA}}}} \right)}} = \left( {\frac{{\nu _{2\;{\mathrm{keV}}} }}{{\nu
_{2500\;{\mathrm{{\AA}}}} }}} \right)^{\alpha _{ox} }  = \,\;403.3^{\alpha _{ox}
} .% (10)
\end{equation}
The definition of $\alpha_{ox}$ used here is slightly different
from that of Zamorani et al. since implicit negative signs are
never used by \Cloudy.
Typical AGN have $\alpha_{ox}\sim  -1.4$.
If X-rays are not present then
$\alpha_{ox} = 0$.

The \cdCommand{ratio} command has an optional keyword,
\cdCommand{alphaox}, which allows $\alpha_{ox}$
to be specified directly.
If the keyword appears then only one parameter
is read, the value of $\alpha_{ox}$.
A generic AGN spectral energy distribution could be produced with
the following,
\begin{verbatim}
blackbody 5e4 K   # the big blue bump
ionization parameter -2
table power law    # an alpha =-1 power law
ratio alphaox -1.4 # set alpha_ox of the X-ray and UV continua
\end{verbatim}
Note that $\alpha_{ox}$ may (or may not) depend on the luminosity of the quasar,
as described by \citet{Avni1986}.
The solid line in their Figure 8 corresponds to
\begin{equation}
\alpha _{ox}  =  - 1.32 - 0.088 \times \log \left( {\frac{{L_o }}{{10^{28}
\,{\mathrm{erg}}\,{\mathrm{s}}^{ - {\mathrm{1}}} \,{\mathrm{Hz}}^{ - {\mathrm{1}}} }}} \right)% (11)
\end{equation}
where they define $L_o$ as the monochromatic optical luminosity
at 2500\AA\  in the source rest frame,
and we assume H$_0 = 50$ and q$_0 = 0$.
Other fits are given
by \citet{Worrall1987}:
\begin{equation}
\alpha _{ox}  =  - 1.11 - 0.111 \times \log \left( {\frac{{L_o }}{{10^{27}
\,{\mathrm{erg}}\,{\mathrm{s}}^{ - {\mathrm{1}}} \,{\mathrm{Hz}}^{ - {\mathrm{1}}} }}} \right)% (12)
\end{equation}
and by \citet{Wilkes1994}:
\begin{equation}
\alpha _{ox}  =  - 1.53 - 0.11 \times \log \left( {\frac{{L_o }}{{10^{30.5}
\,{\mathrm{erg}}\,{\mathrm{s}}^{ - {\mathrm{1}}} \,{\mathrm{Hz}}^{ - {\mathrm{1}}} }}} \right)
.% (13)
\end{equation}
However, \citet{LaFranca1995} find no dependence
of $\alpha_{ox}$ on luminosity.
\citet{Avni1995} find a complicated luminosity dependence.
Clearly this is an area of active research.

\emph{N.B.}  The net incident radiation field may have a
smaller than specified ratio of current
to total incident radiation field,
since the command specifies the ratio of the current
to the previous incident radiation fields,
not the ratio of current to total incident radiation field.
The ionization parameter will be slightly larger than
specified for the same reason.

In general, it is probably better to use the \cdCommand{AGN} command
rather than this command.

This is neither a luminosity nor intensity command---the units
of the previous radiation field carry over to this command.

\section{xi -0.1 }

\citet{Tarter1969, Krolik1981, 2001ApJS..133..221K} define an ionization parameter
$\xi$ given by
\begin{equation}
\xi  =  \left( {4\pi
} \right)^2 \int_{1R}^{1000R}  J_\nu  {\kern 1pt} d\nu /n\left( {\mathrm{H}}
\right)
= \frac{{F_{ion} }}{{n\left( {\mathrm{H}} \right)r^2 }}
\approx \frac{{L_{ion} }}{{n\left( {\mathrm{H}} \right)r^2 }}
 [\mathrm{erg\ cm\ s}^{-1}]
 \label{eqn:xi}
\end{equation}
where $n(\mathrm{H})$ is the hydrogen density at the illuminated face
of the cloud and $r$ is the source - cloud separation.
The number is the log of $\xi$.

The original \citet{Tarter1969} paper defined $\xi$ as the last term in equation \ref{eqn:xi}, 
with $L_{ion}$ defined as including all ionizing radiation.  
\Cloudy\ used this definition, integrating over all ionizing photon energies,
through version C13.03.  

The 1 - 1000 Ryd  energy range first appears in \citet{Krolik1981}, 
in the discussion above eqn 2.2a, 
but they introduce a new variable, $F_{ion}$, for the luminosity over this range.
The XSTAR code \citep{2001ApJS..133..221K} defines  $\xi$ over 1 - 1000 Ryd.

Beginning with C13.04 the XSTAR definition is used for $\xi$,
for compatibility with that code.
That is, we now integrate over 1 - 1000 Ryd.
The middle and right terms in equation \ref{eqn:xi} are not equal for a hard SED.

You can easily define your own energy intervals by using the \cdCommand{luminosity},
page \pageref{sec:LuminosityCommand},
or \cdCommand{intensity}, page \pageref{sec:IntensityCommand}, commands to specify $L$ or $4\pi J$.
They both accept the \cdCommand{range} option to change the limits in equation \ref{eqn:xi}.

