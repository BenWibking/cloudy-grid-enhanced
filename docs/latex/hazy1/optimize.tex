\chapter{THE OPTIMIZE COMMAND}
% !TEX root = hazy1.tex
\label{sec:CommandOptimize}

\section{Running the optimizer}

If you want to run the optimizer, it is {\em mandatory} to use one of the
following commands
\begin{verbatim}
# non-MPI runs
cloudy.exe -r optrun
# alternative
cloudy.exe -p optrun

# MPI runs
mpirun -np <n> /path/to/cloudy.exe -r optrun
# alternative
mpirun -np <n> /path/to/cloudy.exe -p optrun
\end{verbatim}
This assumes that the input script is in the file \cdFilename{optrun.in}. The
normal input and output redirection shown elsewhere will not work in this case
and \Cloudy\ will stop with an error message if you try to use it. The command
line switches are described in a section of
\Hazy~2. Using the \cdCommand{-r} or \cdCommand{-p} command line switches is
necessary because the code needs to know the name of the input script in order
to generate intermediate files with its name embedded. It is also necessary
for the correct functioning of the code in MPI mode. Running in MPI mode is
described further in Section~\ref{sec:OptPhymir}.

\section{Overview}

The \cdCommand{optimize} command and its keywords
tell the code to vary one or more
parameters to try to reproduce e.g.\ an observed emission-line spectrum,
line
flux or luminosity, a set of column densities, or temperatures, etc.
R.F. Carswell wrote the original code and first implemented
the method in \Cloudy\ 
and Peter van Hoof extended the methods.
The user can choose a minimization method (see Section~\ref{sec:optimization_methods}) to obtain a best fit
to a set of
observed quantities.
The latter
are specified by a series
of \cdCommand{optimize} commands.
These are described in Section~\ref{sec:observed_quantities}.
The keyword \cdCommand{vary} can appear on many of the commands
used to specify physical parameters
(Table \ref{tab:CommandWithVaryOption}). This indicates
that this parameter is to be varied in order to obtain a best fit to the observed quantities,
and also sets the initial value for the parameter.
Section~\ref{sec:CommandWithVary_notes}
below discusses details of some commands.
Other commands exist to influence the behavior of the optimizer, e.g. to
set the initial stepsize, limit the range of a physical parameter,
set the convergence tolerance, etc. See Section~\ref{sec:controlling_optimizer} for more details.

\section{Commands with vary option}
\label{sec:CommandsVaryOption}

This section lists all commands which have the \cdCommand{vary} option.
Table \ref{tab:CommandWithVaryOption} lists all commands, and
Section \ref{sec:CommandWithVary_notes} gives notes concerning commands 
with the \cdCommand{vary} option.

\begin{table}
\centering
\caption{\label{tab:CommandWithVaryOption}Commands with vary option. Entries ``var'' are explained in Section~\ref{sec:CommandWithVary_notes}.}
\begin{tabular}{lllll}
\hline
Command& Quantity varied& Min& Max& Inc.\\
\hline
abundances starburst& metallicity& $-3$& $\log(35.5)$& 0.2\\
AGN& Big Bump temperature& $-\infty$& $+\infty$& 0.5\\
blackbody& temperature& $-\infty$& $+\infty$& 0.5\\
bremsstrahlung& temperature& $-\infty$& $+\infty$& 0.5\\
constant pressure set& pressure& $-\infty$& $+\infty$& 0.1\\
constant temperature& temperature& $\log(2.8)$& 10& 0.1\\
coronal equilibrium& temperature& $\log(2.8)$& 10& 0.1\\
cosmic rays& cosmic ray density& $-\infty$& $+\infty$& 0.2\\
database H-like Lyman pumping scale & line blocking scale factor& $-\infty$& $+\infty$& 0.1\\
distance& distance& $-\infty$& $+\infty$& 0.3\\
dlaw& first dlaw parameter& $-\infty$& $+\infty$& 0.5\\
eden& electron density& $-\infty$& 20 & 0.1\\
element \ldots& abundance of an element& $-\infty$& $+\infty$& 0.2\\
energy density& energy density temp& $-\infty$& $+\infty$& 0.1\\
filling factor& filling factor& $-\infty$& 0& 0.5\\
fudge& first fudge factor& $-\infty$& $+\infty$& var\\
globule& density& $-\infty$& $+\infty$& 0.2\\
grains& grain abundance& $-\infty$& $+\infty$& 1.0\\
hden& hydrogen density& $-\infty$& $+\infty$& 1.0\\
hextra& extra heating& $-\infty$& $+\infty$& 0.1\\
illumination& Angle of illumination& 0& $\pi/2$& 0.1\\
intensity& intensity of source& $-\infty$& $+\infty$& 0.5\\
ionization parameter& ionization parameter& $-\infty$& $+\infty$& 0.5\\
luminosity& luminosity of source& $-\infty$& $+\infty$& 0.5\\
magnetic field \ldots& magnetic field& $-\infty$& $+\infty$& 0.5\\
metals& metallicity& $-\infty$& $+\infty$& 0.5\\
metals deplete Jenkins 2009 Fstar 0.5 & Fstar & 0 & 1 & 0.1 \\
phi(H)& photon flux& $-\infty$& $+\infty$& 0.5\\
power law& see below& var& var& 0.2\\
Q(H)& no.\ of ionizing photons& $-\infty$& $+\infty$& 0.5\\
radius& inner radius& $-\infty$& $+\infty$& 0.5\\
ratio& intensity ratio& $-\infty$& $+\infty$& 0.2\\
ratio alphaox& power-law index& $-\infty$& $+\infty$& 0.2\\
set H2 Jura scale& scale factor for Jura rate& $-\infty$& $+\infty$& 0.3\\
set temperature floor& lowest allowed temperature& $\log(2.8)$& 10& 0.3\\
stop \ldots column density& column density& $-\infty$& $+\infty$& 0.5\\
stop mass & total mass& $-\infty$& $+\infty$& 0.5\\
stop optical depth& optical depth at specified energy & $-\infty$& $+\infty$& 0.5\\
stop radius& outer radius& $-\infty$& $+\infty$& 0.5\\
stop thickness& cloud thickness& $-\infty$& $+\infty$& 0.5\\
table Draine& scale factor& $-\infty$& $+\infty$& 0.2\\
table ISM& scale factor& $-\infty$& $+\infty$& 0.2\\
table star \ldots& usually temperature& var& var& 0.1\\
turbulence& turbulent velocity& $-\infty$& $+\infty$& 0.1\\
xi& ionization parameter& $-\infty$& $+\infty$& 0.5\\
\hline
\end{tabular}
\end{table}

\section{What must be specified}

At least one observed quantity must be given, e.g. the desired emission-line spectrum, a line luminosity,
kinetic temperature, or a column density.
You also specify
which physical parameters should be varied.
These parameters are specified by a
keyword \cdCommand{vary} which may appear on any of the
commands listed in
Table \ref{tab:CommandWithVaryOption}.
Up to 20 parameters may be varied at a time.
Note that it makes no sense to vary more parameters than the number of observables.
In that case there will be multiple solutions and the optimizer will not be able to
converge. Preferably you should have significantly more observables than quantities
to vary so that observational errors do not unduly influence the optimization process.
The physical quantities
are actually varied as logs within the code and increments
(the first steps away from the initial guess)
are also logarithmic. The only exceptions to this rule are the
\cdCommand{illumination} command, and commands varying a power-law index
(\cdCommand{power law} and \cdCommand{ratio alphox}).
They are varied as linear quantities.
The parameters of the \cdCommand{dlaw} and \cdCommand{fudge} will
have the meaning given to them by the user and may be either linear or
logarithmic (though the user is strongly encouraged to make them logarithmic).

Several examples of the \cdCommand{vary} option in action
are given in the
\cdFilename{optimize\_*.in} simulations in the test suite.
A typical input stream follows:
\begin{verbatim}
# tell the code to vary the ionization parameter
# and hydrogen density
blackbody, 5e4 K
hden 4 vary
ionization parameter -2 vary
stop zone 1
#
# the following specifies the observed emission lines, order is
# label, wavelength, intensity relative to H-beta, tolerance
optimize lines
O  3 5006.84 intensity =13.8 error =0.1
Blnd 3727 < 2.1 # (only upper limit)
end of lines
#
\end{verbatim}
This example tells the code to vary the density and ionization parameter
to reproduce the observed intensities of two emission lines.

Information concerning the optimization process is fed to the code
as a series of keywords on the \cdCommand{optimize} command.
These are described next.
Only one keyword will be recognized per \cdCommand{optimize} command.

\section{Observed quantities}
\label{sec:observed_quantities}

This section describes how to specify the observed properties we want
to match.

\subsection{Optimize column density}

This specifies a set of column densities to be optimized.
A series of species and constraints are read,
ending with a line with the keyword \cdCommand{end} in columns 1 to 3,
will be read
in from subsequent lines.
One species is entered per line and there is no limit to
the number of entries.

The species can be specified in one of two ways:

\cdTerm{Atomic ions:}
Columns 1 to 4 of the column density lines must
contain the first four characters of the name of the element spelled as
in the output from the zone results.
The first number on the line is the
ionization stage, 1 indicates Atom I, 3 indicates Atom III, etc.
The second
number on the line is the log of the column density [cm$^{-2}$]
and the last
optional number is the relative uncertainty.
It has a default of 0.05 (5
percent).
A column density can be specified as an upper limit by entering
$<$ anywhere on the line after column 4.

The following gives an example of its use:
\begin{verbatim}
optimize column densities
hydrogen 1  < 17 # make optically thin in H^0 Lyman continuum
carbon 4 17.4 error =.001 # match the C+3 column density closely
silicon 3 14.6 # The Si+2 column density
end of column densities
\end{verbatim}

\cdTerm{Molecular and level column densities:}
The species is specified as a quoted, case-sensitive string, and no ionization
stage is specified.  (This is different from previous
versions of Cloudy, where the species did not need to be quoted, but
only the first 4 characters were used, and an ionization stage
of 0 had to be specified.)
The species names which are parsed from the lines giving the column density
constraints
are simply
passed to routine \cdTerm{cdColm}.
That routine is described in Part 2 of \Hazy.
This optimization command accepts any species that
\cdTerm{cdColm} recognizes.
Column densities of molecules and excited states of some ions
can be specified by special syntax, and one of the cases
that is described where \cdTerm{cdColm} is shown in the example below.

The following shows some examples of optimizing molecular and state column
densities:
\begin{verbatim}
optimize column densities
"H-"      < 17 # The H- column density is a limit
"H2"      17.4 error =.001 # Match the total H_2 column density exactly
"H2+"     14.6 # The H2+ molecular ion
"O11*"    16.3 # This is an excited level within the O^0 ground term
"He+[2]"  16.3 # This selects just one level from the He+ ion
"CO[2]"   16.3 # This is a level in the CO molecule
"CO[1:3]" 18.3 # This is the sum of levels 1,2,3 in the CO molecule
end of column densities
\end{verbatim}

\subsection{Optimize, [intensity, luminosity]=36.425 [error =0.1]}

This specifies the luminosity or intensity of a line.
The code will
try to match the predicted intensity or luminosity of the
normalization line
(usually H$\beta$ and set with the \cdCommand{normalize} command)
match this value.
The
sub-keyword is either \cdCommand{intensity} or \cdCommand{luminosity}
and both have exactly the
same effect.
The number is the log of either the intensity or luminosity
of the line in the same units as found in the third column of
the final print out.
The second (optional) number is the fractional error for the
fit between the observed and computed values.
If a tolerance is not
specified then a fractional uncertainty of 0.05 is assumed.

By default the intrinsic luminosity of the line will be matched.
If the keyword \cdCommand{emergent} appears then the
emergent luminosity will be considered.

Comments may be entered by placing any of the special characters
described on page \pageref{sec:CommentsInInput} in column 1.

The following gives some examples of its use:
\begin{verbatim}
# this will request an H beta intensity of 0.5 erg cm^-2 s^-1
# it applies to Hbeta since this is the default normalization
# line
optimize intensity -0.3

# the following resets the normalization line to 5006.84, then
# asks the code to reproduce its luminosity
normalize to "O  3"  5006.84
# we want a 5006.84 luminosity of 10^34.8 erg / s
optimize luminosity 34.8
\end{verbatim}

\subsection{Optimize continuum flux 350 micron 126 Jansky [error =0.1]}
\label{sec:opt:cont:flux}

This commands allows you to optimize an observed continuum flux or surface brightness at an
arbitrary wavelength. Note that both the keywords \cdCommand{continuum} and
\cdCommand{flux} need to be present on the command line. The first parameter
must be the vacuum wavelength or frequency of the observation. The syntax for this
parameter is exactly the same as for the \cdCommand{set nFnu add} command
described in Section~\ref{sec:set:nfnu:add}. The code will implicitly behave
as if a \cdCommand{set nFnu add} command had been given to add the requested
frequency point. The second parameter
on the command line is the observed flux. The unit of the flux should also
be supplied. There are many choices. You can supply any combination of the
following keywords separated by slashes: \cdCommand{erg/s} or \cdCommand{W}
for the power unit, and \cdCommand{sqcm} (cm$^{-2}$) or \cdCommand{sqm}
(m$^{-2}$) for the surface area unit. This will define a flux per decade $\nu
F_\nu$. If you want to supply a flux density $F_\nu$, you can combine this
with the keyword \cdCommand{Hz}, while for $F_\lambda$ you can use \cdCommand{A}
(\AA), \cdCommand{nm} or \cdCommand{micron} ($\mu$m). If you want to supply a surface
brightness you can combine this with \cdCommand{sr} or \cdCommand{sqas} (square arcsec),
otherwise it will be assumed to be the intensity radiated into $4\pi$~sr.
Some examples of valid units would be \cdCommand{erg/s/sqcm/Hz},
\cdCommand{W/sqm/A}, or \cdCommand{W/sqcm/micron/sr}. Alternatively the
following keywords are also recognized: \cdCommand{Jansky} or
\cdCommand{Jy} (Jy), and \cdCommand{mJy} (mJy) for supplying a flux density $F_\nu$,
while the keyword \cdCommand{MJy/sr} can be used for supplying a surface
brightness. A keyword for the unit must be present.
If the flux is $\leq$ 0, or the keyword
\cdCommand{log} appears on the line, it will be assumed that the flux is
logarithmic (but not the wavelength or frequency!). Observed fluxes must be
dereddened for any extinction inbetween the source and the observer. The same
normalization will be used as in the rest of the Cloudy output. So if you
include the \cdCommand{print line flux at earth} command, you should enter the
observed flux at Earth here. See Section~\ref{sec:line:flux:earth} for further
details. See also Section~\ref{sec:set:nfnu} for a discussion of the
components that are included in the continuum flux prediction. The third
(optional) parameter on the command line is the relative uncertainty in the
flux. If it is not specified then an uncertainty of 5\% is assumed. The optimizer
will print both the model and observed flux in the same units as were used on the
command line.
A continuum flux can be specified as an upper limit by entering
$<$ anywhere on the line after column 4.

For wavelengths longer than 1\,mm this command is counted in the
absolute flux category, for shorter wavelengths in the photometry category (see
Section~\ref{sec:conv:criteria}). This distinction is useful so that you can
use a radio continuum flux measurement (e.g.\ at 6~cm) as a replacement for
the H$\beta$ intensity, while you would not want the same to be done for dust
continuum measurements since the latter do not scale with the number of
ionizing photons.

There is no limit to the number of \cdCommand{optimize continuum flux}
commands. The following gives some examples of its use:
\begin{verbatim}
# make sure we normalize the flux as observed
print line flux seen at earth
# and set the distance
distance 980 linear parsec

# optimize a dust continuum flux, e.g. measured by IRAS
# this will be counted in the photometry class
optimize continuum flux 60 micron 220 Jy error 0.10

# optimize a 6 cm radio continuum measurement
# this will be counted in the absolute flux class
optimize continuum flux 6 cm 50 mJy
# an OCRA measurement...
optimize continuum flux 30 GHz 0.1 jansky

# this will set an upper limit of 1e-21 erg s^-1 cm^-2 Hz^-1 (100 Jy) at 100 micron
optimize continuum flux < 100 micron -21 erg/s/sqcm/Hz
\end{verbatim}

\subsection{Optimize lines}

This command tells the code to read a list of observed lines that it
will try to reproduce.
There is no limit to the number of lines that may be entered.
\begin{verbatim}
#
# the following specifies observed emission lines, order is
# label, wavelength, intensity relative to H-beta, tolerance
optimize lines
O  3 5006.84 intensity =13.8 error =0.1
Blnd 3727 < 2.1 # (only upper limit)
O  3 88.3323m 1.2
O  1 145.495m 1.6
H  1 4.65247m  Elow=105291.6  0.33
end of lines
#
\end{verbatim}

One emission line is specified per line and the line must contain
information in a specific order.
The line starts with a species label and wavelength.
Optionally line disambiguation is supported. See Sect.~\ref{sec:SpecifySpectralLines} for details.
The next quantity is the observed relative line intensity.
This will be in the same units as the relative intensities printed at the
end of the calculation.
Intensities are normally given relative to H$\beta$,
but can be changed to other reference lines with the
\cdCommand{normalize} command.
The last (optional) number is the
relative error in the observed line.
If an error is not specified then
a fractional uncertainty of 0.05 (5\%) is assumed.
A line can be specified
as an upper limit by entering $<$ before the intensity.
Nothing may be entered after the last number, except a comment starting with a comment character.
The default is to use the intrinsic line intensities.
If the keyword \cdCommand{emergent} appears than
those will be used instead. This keyword needs to be added
to the \cdCommand{optimize lines} command, as shown below.
\begin{verbatim}
optimize lines emergent
O  3 88.33m 1.2
O  1 145.5m 1.6
end of lines
\end{verbatim}

The emission lines end with a line that has the keyword
\cdCommand{end} in columns 1 to 3.
If this end does not appear correctly then the code will continue
reading lines until the end-of-file is encountered.

Comments may be entered using any of the special characters in column
1 that were described on page~\pageref{sec:CommentsInInput}.

\subsection{Optimize temperature}

This specifies a set of observed mean temperatures.
Each of the lines
that follow gives a species, ion stage, mean temperature, an error,
and a specification of whether to weight the temperature with
respect to radius or volume.
The following is an example:
\begin{verbatim}
optimize temperature
Hydrogen  1  36200K  volume
H2   0   150K     radius
end of temperatures
\end{verbatim}

This behaves as do the previous commands.
The ``$<$'' option is
recognized.
Numbers $\le 10$ are interpreted as logs of the temperature.
The
keywords \cdCommand{volume} and \cdCommand{radius} specify how to
weight the mean temperature.
The default is to weight over radius and will be used if no keyword is
recognized.
The label for the element must appear in the first four columns.
The ion stage follows with the temperature and an optional error coming
next.
An ion stage of zero indicates a molecule.

This command simply copies the labels and parameters on the each line
and passes them to \cdTerm{cdTemp},
a routine that determines the mean temperature
and is described in a section of Hazy 2.
That documentation describes
how to request mean temperatures for various ions, molecules, and derived
quantities.

\subsection{Optimize diameter}

The specifies the (angular) diameter of the nebula, defined either as 2 times
the Str\"omgren radius for an ionization bounded nebula, or 2 times the outer
radius for a density bounded nebula. If the distance to the object is set with
the \cdCommand{distance} command, the number will be interpreted as a linear
diameter in arcsec. If the keyword \cdCommand{cm} is present, the number
will be interpreted as a linear diameter in cm. The keyword \cdCommand{log}
can be used to force interpretation as a logarithmic quantity. If the distance
to the object is not set, the number will always be interpreted as a diameter
in cm.
A diameter can be specified as an upper limit by entering
$<$ anywhere on the line after column 4.

\section{Optimization methods}
\label{sec:optimization_methods}

Two optimization methods are in \Cloudy.
The \cdTerm{phymir} method (\citealp{VanHoof1997}) is the default.

\subsection{Optimize phymir [sequential mode] [8 cpu]}
\label{sec:OptPhymir}

This uses Peter van Hoof's \cdTerm{phymir} optimization method
(\citealp{VanHoof1997}). In particular it tries to avoid being upset by the
inevitable numerical noise that is present in any simulation. It has the
further option of being able to run on multiple CPUs on multiprocessor UNIX
machines, or in an MPI environment. This optimizer is the default.

\cdTerm{Phymir} by default runs in parallel mode (using more than one CPU) on
UNIX systems and in sequential mode on non-UNIX systems. When the code is
compiled with MPI support enabled (see a section of Hazy 2), it will also run
by default in parallel mode on any (distributed) MPI cluster. The keyword
\cdCommand{sequential} forces \cdTerm{phymir} to sequential mode. On most
systems phymir will be able to detect the number of available CPUs. By default
it will use all these CPUs. The command parser searches for a number on the
line. If one is present this will be the maximum number of CPUs that are used
simultaneously. In MPI runs the number of CPUs will be picked up automatically
from the MPI environment. In this case setting the number of CPUs on the
command line will have no effect.

\cdTerm{The continue option}. The \cdTerm{phymir} method includes an option to
restart the optimization. During an optimization run, the code's state is
regularly saved. The optimization can be
restarted using this state file by including the command \cdCommand{optimize
continue} described below.

\subsection{Optimize Subplex}

This uses the \citet{Rowan1990} implementation of the subspace-searching
simplex optimization method. This method does not support the
\cdCommand{continue} option described above.

\section{Controlling the optimizer}
\label{sec:controlling_optimizer}

These commands change the behavior of the optimizer.

\subsection{Optimize file= ``best.in''}
\label{sec:optimize_file}

At the end of the optimization process the derived input parameters are
written into a file for later use.\footnote{This replaces the \cdCommand{optimize save} command, that was in versions 90
and before.}  The default filename is
``\cdFilename{optimal.in}'' but can be changed with this command.
The file name must
be valid for the system in use and must be enclosed in double quotes.

It is possible to use this file to do later calculations in which various
quantities might be saved for plotting.
Also it is generally a good idea
to confirm that a single run with \Cloudy\ does reproduce the same results
as the many calls of the code made by the optimizer.
The two should agree
\emph{exactly} but might not if the code became corrupted
during the many calls
made during the process.

\subsection{Optimize increment = 0.5 dex}

Increments are the logarithmic quantities that will be added to or
subtracted from the initial guess as the optimizer makes the first steps
away from the initial value.
The default increments preset in the code
and listed in Table \ref{tab:CommandWithVaryOption} were chosen with
typical conditions in mind.
It may be necessary to change these if the process is
unable to identify
a solution.
If a zero or no number is entered as an increment then
the default increment will not be changed.

The increments entered with this command affect
\emph{only} the \emph{previously}
entered \cdCommand{vary} command.
The following gives some examples of changing the
increments.
\begin{verbatim}
hden 4 vary
optimize increments .1  # this sets .1 dex changes in hden
brems 6 vary            # increments left at default
radius 13.6 vary
optimize increments .05 # this sets changes in radius
\end{verbatim}

\subsection{Optimize, iterations = 1200}

This specifies the upper limit to the number of iterations to be
performed.
The default is 400.
This command is quite similar to the
\cdCommand{stop zone} command in that it is a fail-safe
method to prevent runaway infinite
loops.
The optimization process should not stop because of this limit.
It may be necessary to increase the limit if the process is still making
progress at the end of the calculation.

The number of iterations that the optimizer needs depends on the number
of parameters that are varied.
To work well there must be enough iterations to allow
each of the free parameters to be varied over a wide range.
Make sure that
this happens by looking at why the optimization process ended.
If it ran
out of iterations then you should check whether all of the parameters been
varied over a significant range.
If they have not then you should increase
the number of iterations or vary fewer parameters.
If you are using the \cdTerm{phymir} method, it is possible to
salvage a prematurely terminated run by increasing the number of
iterations and adding the \cdCommand{optimize continue} command
described below.
\Cloudy\ will then continue where the previous run stopped.

\subsection{Optimize range -2.3 to 3.9}

The preset limits to the range over which parameters can be varied are
indicated in Table \ref{tab:CommandWithVaryOption}.

It is sometimes necessary to establish physical limits to parameters.
For instance metallicities may be limited to the range
$-1 \le \log(\mathrm{Z}) \le 0$ by
observations or physical plausibility.
This command sets the bounds that
the optimizer will use.
The arguments are the lower and upper limit to
the range of variation of the \emph{previously} entered \cdCommand{vary} commands.
Examples follow.
\begin{verbatim}
hden 4 vary
# the following sets limits to range of density
optimize range from 3 to 5
# There will be no range for this one
brems 6 vary
radius 13.6 vary
# this sets limits to radius
optimize range from 13 to 14
\end{verbatim}

If the optimizer tries to go outside of the allowed range, a very large
$\chi^2$ is returned without actually calculating the model. This should
steer the optimizer back to the allowed range.

\subsection{Optimize tolerance = 0.02}

The tolerance is a measure of the desired fractional accuracy of the
parameters that are varied and is set with this command.
The default value
of 0.10 should be sufficient for initial trial runs
but the final values
should be made lower for more precision.

\subsection{Optimize continue}

The \cdTerm{phymir} method includes an option to
restart the optimization. During an optimization run, the code's state is
regularly saved in the file \cdFilename{continue.pmr}. The optimization can be
restarted by including the command \cdCommand{optimize continue}. In this case
it does not start from scratch, but re-starts using information in the file
\cdFilename{continue.pmr} instead. This can be useful when your machine
crashed during a lengthy run, you hit the maximum number of iterations before
the optimization was finished, or if you want to continue optimizing with a
tighter tolerance. The code will be in exactly the same state when it
continues as when it wrote the last state file.
So in general you will need to edit the input script before you
restart the calculation.
For instance, the
first optimization may have ended because it hit the limit to the number of
iterations. If you restart an optimization with a \cdFilename{continue.pmr}
file you will need to increase the maximum number of iterations in the input
script. Otherwise the code would stop immediately, thinking it was already at the
iteration limit.

When you start a run with the \cdCommand{optimize continue} command, the code
will immediately rename the file \cdFilename{continue.pmr} to
\cdFilename{continue.bak} to avoid it being overwritten by the state files
that are produced by the new run. So if you make a mistake, you can rename
\cdFilename{continue.bak} to \cdFilename{continue.pmr} and try again.

The \cdCommand{optimize continue} command is {\em only} supported by the \cdTerm{phymir}
optimizer.

\section{Convergence criteria}
\label{sec:conv:criteria}

Observables are placed into one of six categories - ``relative flux,''
``column density,'' ``mean temperature,'', ``absolute flux'', ``photometry'', and ``diameter''
(see \citealp{VanHoof1997}).
For the $i^{th}$ observable the error estimate is
\begin{equation}
\chi _i^2  = \left( {\frac{{F_i^m  - F_i^0 }}{{\min \left( {F_i^m ,F^0 }
\right)\sigma }}} \right)^2 % (78)
\end{equation}
where $F^m$ and $F^0$ are the model and observed values
and $\sigma$ is the relative
error in the observed value.
The uncertainty $\sigma$ is specified when the
observed quantities are entered and has a default value of 0.05.
The average
of the error estimate is computed for each category.
The error estimate
summed over all categories is minimized.

\section{Other optimizer commands}

\subsection{No vary}

It is sometimes useful to be able to turn off the optimizer for a given
input stream without having to change the (possibly many) occurrences of
the \cdCommand{vary} keyword.
If the \cdCommand{no vary} command is entered
then the \cdCommand{vary} keyword
on the other commands will be ignored and a single model will be computed.

The \cdCommand{no vary} command can also be used with
the \cdCommand{grid} command to tell the code to ignore the \cdCommand{grid} command and \cdCommand{vary}
options.

\subsection{Optimize, trace start 4}

This turns on trace printout for the $n^{th}$ time the
code is called by the
optimizer.
Specific aspects of the trace are still controlled by the
\cdCommand{trace} command.

\subsection{Optimize, trace flow}

This turns on trace to follow the logical flow within the optimizer.

\section{Notes concerning commands with vary option}
\label{sec:CommandWithVary_notes}

The keyword \cdCommand{vary} can appear on the commands in Table
\ref{tab:CommandWithVaryOption}. All other keywords of the command (except
\cdCommand{trace}) are supported (though in some cases there may be an
implicit unit conversion, e.g.\ if you supply a radius in parsec, the
optimizer will vary the equivalent radius in cm). If multiple parameters are
entered on the command line, only the first parameter will be varied while the
rest will be held fixed. The parameter will be varied as a logarithmic
quantity. Any exceptions to these rules will be noted below.

\subsection{AGN}

Only the Big Bump temperature, the first parameter, can be varied.

\subsection{Cosmic rays}

All keywords except \cdCommand{equipartition} are supported.

\subsection{Dlaw}

The first parameter will be varied as entered by the user (i.e.\ no log will
be taken). The user is however strongly encouraged to use logs in the
definition of the command if the parameter is not of order 1.

\subsection{Element}

Only the keywords \cdCommand{scale} and \cdCommand{abundance} are supported.

\subsection{Fudge}

The first parameter will be varied as entered by the user (i.e.\ no log will
be taken). The user is however strongly encouraged to use logs in the
definition of the command if the parameter is not of order 1.

The initial increment is set to 20\% of the initial estimate, or to 1 if the
initial estimate is zero. Use the \cdCommand{optimize increment} command to
override this default.

\subsection{Illumination}

The illumination angle is varied as a linear quantity in radian.

\subsection{Metals}

The \cdCommand{grains} option can be used to vary the
grain abundance with the metallicity. 
The keyword \cdCommand{depletion} is not supported.

\subsection{Power law}

The \cdCommand{vary} keyword appears in three forms,
\cdCommand{vary}, \cdCommand{varyb}, and \cdCommand{varyc}.
If
\cdCommand{vary} appears then the first parameter,
the slope of the power law, is varied.
If \cdCommand{varyb} appears then the second parameter,
the high-energy cutoff temperature in
Kelvin, is varied.
If \cdCommand{varyc} appears then the last parameter, the low-energy
cutoff, is varied.
Only one parameter may be varied at a time. The power law index is
varied as a linear quantity since it is usually negative.
If either of the cutoff energies is varied, the optimizer limits
will be set such that the lower cutoff cannot move beyond the upper
cutoff, or vice versa.

\subsection{Save output}

It is not possible to use some of the save output options while
optimizing.
Save output is designed to save predictions so that they can
be analyzed after the calculation is complete.
It makes little sense to
do this during an optimization run with its hundreds of calculations.  After
the optimization is complete, rerun the optimal simulation with
\cdCommand{save} commands included.

Most \cdCommand{save} options will work during optimization.
Those that read a
series of lines to determine what quantities to output will not work.
This is because the command parser is not used in its usual mode
when commands
are read during an optimization run.

\subsection{Radius}
\label{sec:RadiusVaryOptions}

It is possible to specify the stopping radius or depth on the line but
it is not possible to vary it.
Only the starting radius is varied.


There could be a major source of confusion if the second parameter is
entered and the two numbers are of the same order of magnitude.
The logic
used to interpret the second number is described
on page \pageref{sec:RadiusCommand} above.
If the second number is greater than the first then it is interpreted as
an outer radius; if less than, then the depth.
As a result the
interpretation of the second number can change while the first number is
varied.
It is safer to set an outer radius with the
\cdCommand{stop thickness} or \cdCommand{stop radius} command
rather than using the second number. If you vary the inner radius while
keeping the outer radius fixed with the \cdCommand{stop radius} command,
make sure that you use the \cdCommand{optimize range} command to keep the
inner radius inside the outer radius. Failing to do so may result in premature
termination of the optimization run.

\subsection{Ratio alphox}

The power law index is
varied as a linear quantity since it is usually negative.

\subsection{Stop thickness}

Only one thickness can be specified. Additional parameters for the second
and subsequent iterations will be ignored.

\subsection{Stop radius}

Only one radius can be specified. Additional parameters for the second and
subsequent iterations will be ignored.

If you want to vary the outer radius while keeping the inner radius fixed,
make sure that you use the \cdCommand{optimize range} command to keep the
outer radius outside the inner radius. Failing to do so may result in
premature termination of the optimization run when the optimizer takes the
outer radius inside the inner radius. Alternatively, you can also use a
combination of the \cdCommand{radius} and \cdCommand{stop thickness} commands.

If you want to vary the inner radius as well as the outer radius of a nebula,
always use a combination of the \cdCommand{radius} and \cdCommand{stop
  thickness} commands. Varying the inner and outer radius simultaneously can
lead to premature termination of the run if the optimizer takes the inner
radius outside the outer radius, and is therefore not recommended.

\subsection{Table star \ldots}

The first parameter on the command line will always be varied, even if that is
not the temperature.
The minimum and maximum of the varied parameter depend on the chosen grid and
are set in accordance.

\section{Notes concerning the optimization process}

\subsection{Use physically motivated initial conditions}

The algorithm will not be able to find a solution if one is not physically
possible.
For instance, an observed He~II $\lambda $4686/H$\beta$ intensity
ratio of 0.5
cannot be produced by a 2e4 K blackbody no matter how many other
parameters are varied (it produces no He$^+$-ionizing radiation).
It is
probably necessary to start with parameters in the general area of the
successful model.
When far from the solution, it is also a good idea to
use a large tolerance (using the \cdCommand{optimize tolerance} command)
to stop it
from over-optimizing a bad solution. Then do follow-up runs with a smaller
tolerance once you have a better idea where the optimum solution is.

\subsection{Change the increment size}

The initial increment will often be the largest step ever taken during
the optimization process.
If the initial parameters are far from the
solution then it may be wise to increase the increments.
Depending on the
optimization method used, it may not be possible to find solutions more
than one or two increments away from the initial guess.
If the increments
are too big it may jump over valid solutions.

\subsection{Set physically motivated limits to the variable quantities}

The optimizer driver uses very simple methods and understands surprisingly
little modern astrophysics.
For instance, while trying to reproduce an
observed He II $\lambda$4686/H$\beta$ intensity ratio of 0.5 by varying the temperature
of a blackbody radiator, the algorithm may just as well examine the consequences
of photoionization by a 100 K radiation field.
Physically, it is known
that He II emission only occurs for stars hotter than
$\sim$5e4 K (AGN3 section 2.5), so there is little purpose
in examining temperatures lower than this.
The process will converge more quickly if
reasonable bounds to the range
of the varied quantities are set using
the \cdCommand{optimize range} command.

This advice is dangerous, of course, since you may limit yourself to
solutions close to those you anticipate.
Experiments should also be
performed far from the anticipated solution.

\subsection{Don't give up!}

My experience is that this process works about a quarter of the time.
The problem is that the algorithm can easily home in on a local minimum
which is actually a very bad global solution.
When this occurs the best
idea is to restart the optimization process with a different
set of initial conditions.
Better yet is to start the process with parameters that give
answers known to be close to the solution,
although there is some danger
of limiting the outcome to be what you expect.
Finally, don't be afraid
to use CPU time.

\section{Other optimization methods}

Much of astrophysics involves solving the inverse problem---knowing an
answer (the spectrum) and trying to deduce the question (the conditions
that caused it).
However, optimizing a multi-dimensional function is more an art
than a science.
We would be interested in learning about,
and possibly adopting, other promising efficient optimization methods
(especially if they can be parallelized).
BSD-compatible license code is necessary.

\section{The optimizer test cases}

The suite of test cases that comes with the code includes scripts to
drive each of the optimizers.
These scripts are \cdFilename{optimize\_*.in}.
These were
produced by first running the code at a hydrogen density of 10$^5$~cm$^{-3}$ and
a temperature of 10$^4$ K.
The spectrum of [\oii] and [\oiii] emission lines
was taken from this calculation.
Each optimizer starts at a density and
temperature some distance away from this solution and tries to reproduce
the spectrum.
Table \ref{tab:OptimizerResults} shows the results of this test.

\begin{table}
\label{tab:OptimizerResults}
\centering
\caption{Optimizer results}
\begin{tabular}{cccc}
\hline
Method& Iterations& Density& Temperature\\
Initial
condition& ---& 5& 4\\
% done with r3246 of the mainline, and g++ 4.3.1 on AMD64 (optimized)
Phymir& 50& 4.988& 3.998\\
Subplex& 64& 5.001& 3.997\\
\hline
\end{tabular}
\end{table}



