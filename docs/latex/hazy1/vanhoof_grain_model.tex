\chapter{USING THE GRAIN CODE IN CLOUDY}
\label{grain:appendix}
% !TEX root = hazy1.tex

\section{Introduction}

This release of Cloudy contains a grain model that constitutes a
significant upgrade from the original model that was described in
\citet{Baldwin1991}. The code has been written by Peter van Hoof, except
where indicated differently. These are the main features:
\begin{itemize}
\item The code resolves the size distribution of the grains in an arbitrary
  number of size bins (chosen by the user), and calculates all grain
  parameters such as temperature, charge distribution, emitted flux, etc.,
  separately for each bin. Since grain opacities depend strongly on the grain
  radius, they need to be calculated separately for each bin as well.
\item For this purpose Cloudy contains a Mie code for spherical grains
  (written by P.G. Martin, based on a code by \citet{Hansen1974}). This code
  allows the user to calculate grain opacities using either a pre-defined or a
  user-supplied set of optical constants or opacities, an arbitrary size
  distribution, and an arbitrary number of grain size bins. The code can
  handle any number of grain bins, only limited by the memory on your machine.
\item The code fully treats quantum heating of grains using a robust and
  efficient algorithm (which is a comprehensively upgraded version of a code
  originally written by K. Volk), implementing an improved version of the
  procedure described in \citet{Guhathakurta1989}. Combined with resolved size
  distributions, this will lead to a much more realistic modeling of the grain
  emission under all circumstances. Quantum heating is turned on automatically
  for all resolved size distributions\footnote{The actual criterion is that
    the ratio of the volumes of the largest and smallest grain in each bin is
    smaller than 100 for quantum heating to be the default. Note that this
    criterion is very generous and that a much smaller ratio (i.e., a higher
    number of size bins) will usually be necessary to achieve proper
    convergence of the emitted spectrum.} and single-sized grains (including
  PAH's, but excluding grey grains for which quantum heating is never used).
  The user can change the default behavior of the code by including the
  keyword \cdCommand{qheat} (to enable quantum heating) or \cdCommand{no
    qheat} (to disable quantum heating) with the \cdCommand{grains} command.
  If the relative grain abundance $A_{\rm rel}(r)$ in a particular zone
  (variable \cdVariable{gv.bin[nd]-$>$dstAbund}) drops below a preset
  threshold value (variable \cdVariable{gv.dstAbundThreshold}, default value
  $10^{-5}$), quantum heating will be disabled for that zone only. Note that
  quantum heating usually only influences the energy distribution of the
  emitted spectrum, nothing else\footnote{Quantum heating can influence the
    emission line spectrum as well if continuum pumping of the lines by the
    short wavelength end of the dust spectrum is important. Typically these
    will be molecular lines.}. So switching off quantum heating can be a good
  idea to speed up the modeling, provided the correct shape of the emitted
  spectrum has no influence on the results the user is interested in.
\item The treatment of the grain physics has been completely overhauled,
  following the discussion in \citet{Weingartner2001a}, \citet{VanHoof2004},
  and \citet{Weingartner2006}. The grain model has a detailed treatment of the
  photo-electric effect and collisional processes, and includes thermionic
  emissions. The code uses the $n$-charge state model where for each bin the
  charge distribution is resolved in exactly $n$ discrete charge states,
  independent of the grain size. The default value for $n$ is 2, but the user
  can choose any number between 2 and \cdVariable{NCHS}/3 = 10 using the
  \cdCommand{set nchrg $<$n$>$} command (note that there is no ``a'' in
  \cdCommand{nchrg}!). Choosing a higher value will usually give more accurate
  results at the expense of computing time. Using the default $n = 2$ should
  give sufficient accuracy for most astronomical applications. The maximum
  value of $n$ can be increased by altering the definition of
  \cdVariable{NCHS} in \cdFilename{grainvar.h} and recompiling the code. It is
  however extremely unlikely that a value of $n > 10$ would ever be needed. A
  detailed description of the $n$-charge state model can be found in
  \citet{VanHoof2004}.

  Extensive comparisons in collaboration with Joe Weingartner done in 2001
  show that the photo-electric heating rates and collisional cooling rates
  predicted by Cloudy agree very well with the results from the
  \citet{Weingartner2001a} model for a wide range of grain sizes (between
  5~\AA\ and 0.1~$\mu$m), and using various choices for the incident radiation
  field. A detailed discussion of this comparison can be found in van
  \citet{VanHoof2001}.
\end{itemize}

\section{Using the grain model}
\label{grain:compile}
In order to use the grain model, two steps need to be taken. In the first
step, the grain opacities are calculated using the \cdCommand{compile grain}
command. In the second step these opacities can be used with the
\cdCommand{grains} command to create the actual photo-ionization model. The
Cloudy distribution comes with a number of pre-compiled opacity files which
cover a number of standard combinations of grain materials and size
distributions. If these are sufficient for your needs, you can skip the skip
the first step, and use these opacities directly in the \cdCommand{grains}
command. If you wish to use different grain materials and/or size
distributions, you will have to compile the grain opacities first with the
\cdCommand{compile grain} command described on page \pageref{sec:CompileGrains}.
In order to do this, you type in a single command line, e.g.:

\begin{verbatim}
compile grain "silicate.rfi"  "ism.szd" 10
\end{verbatim}

\noindent followed by an extra carriage return. The Mie code needs refractive
index or mixed medium, and size distribution files as input. These are files
with names ending in ``\cdFilename{.rfi}'', ``\cdFilename{.mix}'', or
``\cdFilename{.szd}'', resp. It will generate an opacity file, ending in
``\cdFilename{.opc}''. The command shown above will instruct the Mie
code to calculate opacities using optical constants for astronomical silicate
and a \citet{Mathis1977} standard ISM size distribution. The size
distribution will be resolved in 10 bins. This will produce a file
\cdFilename{silicate\_ism\_10.opc} which contains all the opacities. This
opacity file is in human readable form and contains many comments to clarify
its contents. It also contains a table of the size distribution function $a^4
n(a)$ as a function of the grain radius $a$ for reference. The format of the
table is such that it could be used directly to define a size distribution
table (see section~\ref{sdtable}). Note that only a single \cdCommand{compile
  grain} command can be given in a single Cloudy run. Note furthermore that
the opacity file produced in the example above is already included in the
Cloudy distribution!

In the previous example a refractive index file was used to define the optical
properties of a pure grain material (e.g., astronomical silicate or graphite).
A second example could be:

\begin{verbatim}
compile grain "fluffy.mix"  "ism.szd" 10
\end{verbatim}

\noindent (the mixed medium file \cdFilename{fluffy.mix} is
included in the Cloudy distribution, and the text is also shown in
Table~\ref{mixfile}). Here the mixed medium file is used instead of a
refractive index file to define a mixture of two or more grain materials
(possibly including vacuum for defining fluffy grains). In this case the
optical properties of the constituting materials are combined with a mixing
law (also called effective medium theory or EMT) to create a set of optical
constants that represents the mixture and can be fed into the Mie theory.
Hence the refractive index and mixed medium files play similar roles in that
both define a set of optical constants to be used by the Mie code. They differ
in the fact that refractive index files are used to define pure materials,
while mixed medium files define impure materials. As a consequence the
internal format of both files is very different and will be described in more
detail in Sections~\ref{rfi:descr} and \ref{mix:descr}. The opacities
generated from both types of input files are fully compatible though and can
be used in exactly the same way, as shown below.

In the second step, you can use these opacities in a subsequent run of Cloudy
with the \cdCommand{grains} command described in Hazy. An example could be:

\begin{verbatim}
set nchrg <n> # optional
grains "silicate_ism_10.opc" +0.100 log # quantum heating on
grains "fluffy_ism_10.opc" no qheat # quantum heating off
\end{verbatim}

Several \cdCommand{grains} commands may be used simultaneously to define
mixtures of grains. One can freely mix \cdCommand{set nchrg} and
\cdCommand{grains} commands, and different grain types may be calculated using
a different number of charge states. The \cdCommand{set nchrg} command will
only affect \cdCommand{grains} commands that come later in the input file. For
ease of use, the filenames of certain refractive index, size distribution, and
opacity files in the standard Cloudy distribution may be replaced by keywords,
as described in Hazy.

% say something about choice of nBin, convergence of spectrum and GrGH !!

The refractive index, mixed medium, and size distribution files may be
replaced by user-defined versions, giving the user considerably more freedom
to define grain properties compared to the old grain model in Cloudy. The
format for each of those files will be defined in Sections~\ref{rfi:descr},
\ref{mix:descr}, and \ref{szd:descr}.

\section[Description of the refractive index files]
{Description of the refractive index files\protect\footnote{Note that the
term ``refractive index file'' is used rather loosely here. It also pertains
to materials for which no refractive index data in the strict sense exist,
such as grey grains and PAH's.}}
\label{rfi:descr}

In order for the Mie code in Cloudy to work, it needs to know the optical
properties of the grains under consideration. If the grain consists of a
single, pure material, these have to be defined in a separate file with a name
that must end in ``\cdFilename{.rfi}''. In this section we will describe the
format of this file. It is helpful to compare with e.g. the
\cdFilename{graphite.rfi} or \cdFilename{silicate.rfi} file in the standard
distribution while reading this section. This document pertains to refractive
index files with magic number \cdVariable{1030103}. Mixtures of grain
materials can also be defined using mixed medium files which are described in
Section~\ref{mix:descr}.

As is the case with all files connected with the Mie code in Cloudy, the user
has the freedom to add comments to the file provided they start with a sharp
sign (\#). These comments may either occupy an entire line (in which case the
sharp sign has to be in the first column), or be appended to some input value.
Comments have been liberally added to the refractive index files that come
with the standard Cloudy distribution in an effort to make them
self-explanatory. All refractive index files start with a magic number for
version control. This number should simply be copied from the files in the
standard distribution. Next comes the chemical formula of the grain material,
for graphite this would simply be ``\cdMono{C}''; for a certain type of
silicate this could be ``\cdMono{Mg0.6Fe0.4SiO3}'' indicating
Mg$_{0.6}$Fe$_{0.4}$SiO$_3$. Note that the formula is case sensitive. For
simplicity I will call this elementary building block the grain molecule, even
though this term is not always appropriate. The molecular weight will be
calculated by Cloudy using the chemical formula. The next two lines in the
refractive index file define the default abundance of the grain molecule. The
first number gives the maximum number density $A_{\rm max}$ of the grain
molecule (relative to hydrogen = 1) that can be formed, assuming it completely
depletes at least one of the constituting atoms from the gas phase. Let us
assume that the initial abundances in the gas phase (i.e., abundances {\em
  before} grains were formed) were $A(X)$. Then , for the silicate example
above, $A_{\rm max}$ should be min($A({\rm Mg})/0.6$, $A({\rm Fe})/0.4$,
$A({\rm Si})$, $A({\rm O})/3$). The second number gives the fraction $A_{\rm
  eff}$ of the maximum amount that is actually formed (i.e., the efficiency of
the process), and should be a number between 0 and 1. The default abundance of
the grain molecule is then given by the product of these two numbers: $A_{\rm
  eff} A_{\rm max}$. The actual grain abundance used in the Cloudy modeling
can be set with the \cdCommand{grains} and the \cdCommand{metals} command (see
Hazy for details). This essentially defines an additional multiplier $A_{\rm
  rel}$ which may be either smaller or larger than 1, and may depend on
position $r$. The actual grain molecule abundance used in the Cloudy model is
then given by $A_{\rm rel}(r) A_{\rm eff} A_{\rm max}$. For the silicate
example above, the number density of iron locked up in these grains would be
given by $0.4 A_{\rm rel}(r) A_{\rm eff} A_{\rm max}$ (relative to hydrogen =
1), or $0.4 A_{\rm rel}(r) A_{\rm eff} A_{\rm max} n_{\rm H}(r)$ (in
atoms/cm$^3$, $n_{\rm H}$ is the hydrogen number density as defined by the
\cdCommand{hden} command). The next line in the refractive index file gives
the specific density of the grain material in g/cm$^3$. The grain code needs
to know the material type in order to determine certain grain properties that
are currently hardwired into the code. Examples would be the grain enthalpy as
a function of temperature and the ro-vibrational distribution of H$_2$ formed
on the grain surface. The next line indicates which material type to use.
Currently six choices are available. They are outlined in
Table~\ref{mat:type}. The next two lines give the work function and the
bandgap between the valence and conduction band in Rydberg respectively. For
conductors the bandgap should be set to zero, for insulators such as
silicates, a non-zero value should be used. The next line gives the efficiency
of thermionic emissions. This parameter is usually unknown for materials of
astrophysical interest, and using 0.5 should be a reasonably safe guess. Next
comes the sublimation temperature in Kelvin. Then comes a line with a keyword
which identifies what type of refractive index file is being read. It
determines what the remainder of the file will look like. Allowed values are
\cdVariable{rfi\_tbl}, \cdVariable{opc\_tbl}, \cdVariable{grey},
\cdVariable{pah1}, \cdVariable{ph2n}, \cdVariable{ph2c}, \cdVariable{ph3n},
and \cdVariable{ph3c}. In the \cdVariable{grey}, \cdVariable{pah1},
\cdVariable{ph2n}, \cdVariable{ph2c}, \cdVariable{ph3n}, and \cdVariable{ph3c}
cases, no further data is needed and the refractive index file ends there
(i.e., the information needed to calculate the opacities is hardwired in the
code).

\begin{table}
\caption[The definition of each of the material types hardwired into Cloudy.]
{The definition of each of the material types hardwired into Cloudy.
The code in column 1 needs to be entered in the refractive index file. Each of
the entries in columns 3 and higher will be explained in the table indicated
in the header of that column.}
\vspace*{-6mm}
\label{mat:type}
\begin{center}
\footnotesize
\begin{tabular}{rllllllll}
\hline
code & mnemonic & Table~\ref{enthalpy} & Table~\ref{zmin} & Table~\ref{affinity} & Table~\ref{ial} & Table~\ref{pe} & Table~\ref{storage} & Table~\ref{htwo:rate} \\
\hline
1 & \cdVariable{MAT\_CAR}  & \cdVariable{ENTH\_CAR}  & \cdVariable{ZMIN\_CAR} & \cdVariable{POT\_CAR} & \cdVariable{IAL\_CAR} & \cdVariable{PE\_CAR} & \cdVariable{STRG\_CAR} & \cdVariable{H2\_CAR} \\
2 & \cdVariable{MAT\_SIL}  & \cdVariable{ENTH\_SIL}  & \cdVariable{ZMIN\_SIL} & \cdVariable{POT\_SIL} & \cdVariable{IAL\_SIL} & \cdVariable{PE\_SIL} & \cdVariable{STRG\_SIL} & \cdVariable{H2\_SIL} \\
3 & \cdVariable{MAT\_PAH}  & \cdVariable{ENTH\_PAH}  & \cdVariable{ZMIN\_CAR} & \cdVariable{POT\_CAR} & \cdVariable{IAL\_CAR} & \cdVariable{PE\_CAR} & \cdVariable{STRG\_CAR} & \cdVariable{H2\_CAR} \\
4 & \cdVariable{MAT\_CAR2} & \cdVariable{ENTH\_CAR2} & \cdVariable{ZMIN\_CAR} & \cdVariable{POT\_CAR} & \cdVariable{IAL\_CAR} & \cdVariable{PE\_CAR} & \cdVariable{STRG\_CAR} & \cdVariable{H2\_CAR} \\
5 & \cdVariable{MAT\_SIL2} & \cdVariable{ENTH\_SIL2} & \cdVariable{ZMIN\_SIL} & \cdVariable{POT\_SIL} & \cdVariable{IAL\_SIL} & \cdVariable{PE\_SIL} & \cdVariable{STRG\_SIL} & \cdVariable{H2\_SIL} \\
6 & \cdVariable{MAT\_PAH2} & \cdVariable{ENTH\_PAH2} & \cdVariable{ZMIN\_CAR} & \cdVariable{POT\_CAR} & \cdVariable{IAL\_CAR} & \cdVariable{PE\_CAR} & \cdVariable{STRG\_CAR} & \cdVariable{H2\_CAR} \\
7 & \cdVariable{MAT\_SIC}  & \cdVariable{ENTH\_SIC} & \cdVariable{ZMIN\_CAR} & \cdVariable{POT\_CAR} & \cdVariable{IAL\_CAR} & \cdVariable{PE\_CAR} & \cdVariable{STRG\_CAR} & \cdVariable{H2\_CAR} \\
\hline
\end{tabular}
\end{center}
\end{table}

\begin{table}[p]
\caption{The various choices for the enthalpy function hardwired in Cloudy.}
\label{enthalpy}
%\begin{center}
\small
\begin{tabular}{lll}
\hline
mnemonic & type & reference \\
\hline
\cdVariable{ENTH\_CAR}  & graphite & \citet{Guhathakurta1989}, Eq.~3.3 \\
\cdVariable{ENTH\_SIL}  & silicate & \citet{Guhathakurta1989}, Eq.~3.4 \\
\cdVariable{ENTH\_PAH}  & PAH & \citet{Dwek1997}, Eq.~A4 \\
\cdVariable{ENTH\_CAR2} & graphite & \citet{Draine2001}, Eq.~9 \\
\cdVariable{ENTH\_SIL2} & silicate & \citet{Draine2001}, Eq.~11 \\
\cdVariable{ENTH\_PAH2} & PAH & \citet{Draine2001}, Eq.~33 \\
\cdVariable{ENTH\_SIC}  & $\alpha$-SiC & \citet{Chekhovskoy1971} \\
\hline
\end{tabular}
%\end{center}
\end{table}

\begin{table}[p]
\caption{The various choices for the minimum charge hardwired in Cloudy.}
\label{zmin}
%\begin{center}
\small
\begin{tabular}{lll}
\hline
mnemonic & type & reference \\
\hline
\cdVariable{ZMIN\_CAR}  & graphite & \citet{Weingartner2001a}, Eq.~23a+24 \\
\cdVariable{ZMIN\_SIL}  & silicate & \citet{Weingartner2001a}, Eq.~23b+24 \\
\hline
\end{tabular}
%\end{center}
\end{table}

\begin{table}[p]
\caption{The various expressions for the electron affinity hardwired in Cloudy.}
\label{affinity}
%\begin{center}
\small
\begin{tabular}{lll}
\hline
mnemonic & type & reference \\
\hline
\cdVariable{POT\_CAR}  & graphite & \citet{Weingartner2001a}, Eq.~4 \\
\cdVariable{POT\_SIL}  & silicate & \citet{Weingartner2001a}, Eq.~5 \\
\hline
\end{tabular}
%\end{center}
\end{table}

\begin{table}[p]
\caption[The various choices for the inverse attenuation length.]
{The various choices for the inverse attenuation length. Note that
these choices will {\em only} be used if no refractive index data is contained
in the refractive index file.}
% \vspace{1mm}
\label{ial}
%\begin{center}
\small
\begin{tabular}{lll}
\hline
mnemonic & type & description \\
\hline
\cdVariable{IAL\_CAR}  & graphite & use \cdFilename{graphite.rfi} to calculate the inverse attenuation length \\
\cdVariable{IAL\_SIL}  & silicate & use \cdFilename{silicate.rfi} to calculate the inverse attenuation length \\
\hline
\end{tabular}
%\end{center}
\end{table}

\begin{table}[p]
\caption{The various expressions for the photoelectric yield hardwired in Cloudy.}
\label{pe}
%\begin{center}
\small
\begin{tabular}{lll}
\hline
mnemonic & type & reference \\
\hline
\cdVariable{PE\_CAR}  & graphite & \citet{Weingartner2006}, Sect.\ 4 -- 6 \\
                      & graphite & \citet{Weingartner2001a}, Eq.~16 (with \cdCommand{no grain x-ray treatment}) \\
\cdVariable{PE\_SIL}  & silicate & \citet{Weingartner2006}, Sect.\ 4 -- 6 \\
                      & silicate & \citet{Weingartner2001a}, Eq.~17 (with \cdCommand{no grain x-ray treatment}) \\
\hline
\end{tabular}
%\end{center}
\end{table}

\begin{table}[p]
\caption{The various choices for splitting up the grain emissions.}
\label{storage}
%\begin{center}
\small
\begin{tabular}{lll}
\hline
mnemonic & type & description \\
\hline
\cdVariable{STRG\_CAR}  & graphite & store emitted spectrum as graphitic emission. \\
\cdVariable{STRG\_SIL}  & silicate & store emitted spectrum as silicate emission. \\
\hline
\end{tabular}
%\end{center}
\end{table}

\begin{table}[p]
\caption{The expressions for the ro-vibrational distribution of H$_2$ formed on various grain surfaces.}
\label{htwo:rate}
%\begin{center}
\small
\begin{tabular}{lll}
\hline
mnemonic & type & reference \\
\hline
\cdVariable{H2\_ICE}  & ice mantle & \citet{Takahashi2001b}, Table 2 \\
\cdVariable{H2\_CAR}  & graphite   & \citet{Takahashi2001b}, Table 2 \\
\cdVariable{H2\_SIL}  & silicate   & \citet{Takahashi2001b}, Table 2 \\
\hline
\end{tabular}
%\end{center}
\end{table}

\subsection{rfi\_tbl}

This format is used to define the refractive index as a function of wavelength,
which will then be used by the Mie code to generate the opacities when combined
with a size distribution file.

The remaining lines in the refractive index file define the optical constants.
First the user has to enter a code to define how the complex refractive index
$n$ is written up; the following choices are supported:
\begin{enumerate}
\item
-- supply Re($n^2$), Im($n^2$) (the dielectric function),
\item
-- supply Re($n-1$), Im($n$),
\item
-- supply Re($n$), Im($n$).
\end{enumerate}
The next line gives the number of principal axes $N_a$ for the grain crystal
and should be a number between 1 and 3. For amorphous materials one should
always choose 1 axis. For crystalline materials the number may be 2 or 3. Next
comes a line with $N_a$ numbers giving the relative weights for each of the
axes. These numbers will be used to average the opacities over each of the
axes in crystalline materials. For materials with only one axis, this number
is obviously redundant, and a single 1 should be entered. For graphite it is
appropriate to enter ``1~2'' indicating that the first axis will have relative
weight 1/3 and the second 2/3 (i.e., the relative weights are 1:2). Next come
$N_a$ chunks of data defining the optical constants for each axis. Each chunk
starts with a line giving the number of data points $N_d$ for that axis,
followed by $N_d$ lines containing 3 numbers: the wavelength in $\mu$m, and
the real and imaginary part of the complex number defined above. Note that the
wavelengths may be either monotonically increasing or decreasing, and that the
number of data points or the wavelength grid may be different for each axis.
The grid of wavelengths need not coincide with the Cloudy grid, nor does it
need to cover the entire range of energies used in Cloudy. Logarithmic
interpolation or extrapolation will be performed where needed. {\em It is the
users responsibility to ensure that each wavelength grid contains sufficient
points to make this process meaningful; only minimal checks will be performed
by Cloudy.} Note that if your refractive index data do not have sufficient
wavelength coverage (as will usually be the case if they come from laboratory
experiments) you cannot simply merge data from another refractive index file
to extend the range. This is because the refractive index data need to obey
the Kramers-Kronig relations. If you take data from two different sources,
this will not be the case. In most cases you need to extend your data into the
EUV and X-ray regime. The proper procedure in this case is to calculate the
absorption opacity by adding the photoionization opacities (including
inner-shell ionizations) for each of the atoms that make up the grain using
data for neutral atoms. You then need to combine these data with the
Kramers-Kronig relations to reverse engineer the refractive index data. This
is the procedure that was used to create the refractive index files that are
included in the Cloudy distribution.

\subsection{opc\_tbl}

This format can be used to directly define the opacities as a function of
photon energy. It is mainly useful to define alternative prescriptions for
PAH's. Note that the refractive index file created this way still needs to be
combined with a size distribution file in the usual way with the
\cdCommand{compile grain} command. This step will not alter the opacities
themselves, but is necessary to compute certain quantities that are related to
the size distribution which are not defined in the refractive index file
itself. It is not allowed to resolve the size distribution when using an
opacity table. If you want to define a resolved size distribution, you will
have to supply an opacity table and corresponding size distribution file for
each size bin separately.

The remaining lines in the refractive index file define the opacities. The
first line contains the number of data values $N_v$ supplied on each line in
the table. Allowed values are:
\begin{enumerate}
\item
-- supply $\sigma_{\rm abs}$ only (in cm$^2$/H),
\item
-- supply $\sigma_{\rm abs}$, $\sigma_{\rm sct} \times (1-g)$ (in cm$^2$/H),
\item
-- supply $\sigma_{\rm abs}$, $\sigma_{\rm sct}$ (in cm$^2$/H), and $(1-g)$
   separately.
\item
-- supply $\sigma_{\rm abs}$, $\sigma_{\rm sct}$ (in cm$^2$/H), $(1-g)$,
   and the inverse attenuation length (in cm$^{-1}$).
\end{enumerate}
Here $\sigma_{\rm abs}$ and $\sigma_{\rm sct}$ are the absorption and
scattering cross sections, and $g$ is called the asymmetry factor or phase
function. The latter is needed since the radiative transfer in Cloudy assumes
that photons that are scattered in a forward direction (at an angle of less
than 90$^\circ$) are not lost from the beam (i.e., they still make it out of
the cloud). The asymmetry factor $g$ gives the average value of the cosine of
the scattering angle, so that $(1-g)$ is an approximate correction factor for
the anisotropy of the scattering by grains. If no scattering cross sections
are supplied, $\sigma_{\rm sct} \times (1-g)$ will default to 10\% of the
absorption cross section. If no asymmetry factor is supplied, $g$ will default
to zero. If no inverse attenuation lengths are supplied, they will be derived
from the material type supplied previously in the file. On the next line is a
keyword with two allowed values: \cdCommand{log} or \cdCommand{linear},
defining whether the opacity table will contain logarithmic or linear data
values. The next line gives the number of data points $N_d$, followed by $N_d$
lines of actual data. Each line of the table should contain $N_v+1$ numbers,
starting with the photon energy in Rydberg, followed by the absorption cross
section, the scattering cross section, the factor $(1-g)$, and the inverse
attenuation length (the last three values may be omitted, depending on the
value of $N_v$). The photon energies must either be strictly monotonically
increasing or decreasing. The grid of photon energies need not coincide with
the Cloudy grid, nor does it need to cover the entire range of energies.
Logarithmic interpolation or extrapolation will be performed where needed.
{\em It is the users responsibility to ensure that the frequency grid contains
  sufficient points to make this process meaningful; only minimal checks will
  be performed by Cloudy. }

\section{Description of the mixed medium files}
\label{mix:descr}

In order to define the optical properties of a grain consisting of a mixture
of materials\footnote{In this section the term material can also mean vacuum
  so that the user can define fluffy grains by mixing one or more materials
  with vacuum.}, a separate file with a name that must end in
``\cdFilename{.mix}'' has to be used. In this section we will describe the format
of this file. An example of a mixed medium file is shown in
Table~\ref{mixfile}. This document pertains to mixed medium files with magic
number \cdVariable{4030103}.

\begin{table}
\caption[Example of a mixed medium file for fluffy silicate.]
{Example of a mixed medium file for fluffy silicate (\cdFilename{fluffy.mix}).}
\label{mixfile}
\begin{verbatim}
# toy model of fluffy silicate; for test purposes only!
4030103  # magic number for version control
0.61     # default depletion
2        # number of separate materials in grain
#
  50  "vacuum.rfi"
  50  "silicate.rfi"
#
br35     # which EMT should be used
\end{verbatim}
\end{table}

As is the case with all files connected with the Mie code in Cloudy, the user
has the freedom to add comments to the file provided they start with a sharp
sign (\#). These comments may either occupy an entire line (in which case the
sharp sign has to be in the first column), or be appended to some input value.
%Comments have been liberally added to the refractive index files that come
%with the standard Cloudy distribution in an effort to make them
%self-explanatory.
All mixed medium files start with a magic number for version control, as shown
in Table~\ref{mixfile}.
%This number should simply be copied from the files in the
%standard distribution.
The chemical formula and the abundance at maximum depletion of the mixed
material will be derived by Cloudy using the information supplied in the
refractive index files of the constituting materials. However, the default
depletion cannot be calculated and needs to be supplied by the user on the
next line. This number is defined as a fraction of the maximum depletion, and
should be a number between 0 and 1 (see also the discussion in
Section~\ref{rfi:descr}). When the opacities are calculated, the code will
print the chemical formula and the abundances (both at maximum and default
depletion), so that the user can check whether the default abundance is
correct. The next line gives the number $n$ of separate materials in the grain
($n$ should be at least 2; it is allowed to specify the same material several
times), followed by $n$ lines giving information about the materials. Each of
these lines should give the relative fraction of the volume occupied by that
particular material, followed by the name of the refractive index file between
double quotes. Note that the refractive index files have to be of type
\cdVariable{rfi\_tbl} since mixing laws need optical constants as input and
cannot work with opacities. The normalization of the relative volumes can be
on an arbitrary scale. The last line in the file should be a keyword
identifying which mixing law is to be used. Allowed values are listed in
Table~\ref{mixid}. In the case of \citet{Farafonov2000}, the mixing law
assumes that the grain consists of concentric layers of material. It is
assumed that the first material supplied in the mixed medium file identifies
the innermost layer, and the following lines identify the subsequent layers
towards the outer edge of the grain. The outermost layer will determine the
material type as defined in Table~\ref{mat:type}. None of the layers are
allowed to be crystalline, i.e., have more than one principal axis, and the
outermost layer is also not allowed to be vacuum. In the case of
\citet{Bruggeman1935} and \citet{Stognienko1995}, the grain is assumed to be a
random (possibly fractal) mixture of materials and the sequence in the mixed
medium file is irrelevant, except that the material type of the mixture is
still defined by the last entry in the list of materials. For this reason it
is not allowed to have vacuum as the last entry. In the case of
\citet{MaxwellGarnett04} the grain is assumed to consist predominantly of a
matrix which is defined by the first entry in the mixed medium file, which
encapsulates small, random inclusions which are defined by the remaining
entries. The material type, as well as many other properties of the grain will
be determined by the matrix material, i.e., the first entry in the file. The
matrix material is not allowed to be vacuum or a crystalline material, but the
inclusions can be.

\begin{table}
\caption{Allowed choices for the mixing law.}
\label{mixid}
\begin{tabular}{ll}
\hline
mnemonic & reference \\
\hline
\cdVariable{BR35} & \citet{Bruggeman1935} \\
\cdVariable{FA00} & \citet{Voshchinnikov1999} \\
                  & \citet{Farafonov2000} \\
\cdVariable{MG04} & \citet{MaxwellGarnett04} \\
\cdVariable{ST95} & \citet{Stognienko1995} \\
\hline
\end{tabular}
\end{table}

\section{Description of the size distribution files}
\label{szd:descr}

In order for the Mie code in Cloudy to work, it needs to know the size
distribution of the grains under consideration. This distribution has to be
defined in a separate file with a name that must end in ``\cdFilename{.szd}''.
In this section I will describe the format of this file. This document
pertains to size distribution files with magic number \cdVariable{2010403}.

If we denote the number of grains $n_{\rm g} d a$ with radii between $a$ and
$a + d a$ as $n_{\rm g} d a = n(a)d a$, the purpose of the size distribution
file is to define $n(a)$, or alternatively $a^4 n(a)$ which is more commonly
used.

As is the case with all files connected with the Mie code in Cloudy, the user
has the freedom to add comments to the file provided they start with a sharp
sign (\#). These comments may either occupy an entire line (in which case the
sharp sign has to be in the first column), or be appended to some input value.
Comments have been liberally added to the size distribution files that come
with the standard Cloudy distribution in an effort to make them
self-explanatory. All size distribution files start with a magic number for
version control. This number should simply be copied from the files in the
standard distribution. The next line should contain a keyword indicating which
type of size distribution will be entered. The following choices are currently
supported: \cdVariable{ssize} - a single sized grain, \cdVariable{ncarb} - a
PAH molecule with a specified number of carbon atoms, \cdVariable{power} - a
simple power law, \cdVariable{exp1}, \cdVariable{exp2}, \cdVariable{exp3} -
power laws with an exponential cutoff, \cdVariable{normal},
\cdVariable{lognormal} - a Gaussian distribution in $a$ or $\ln(a)$,
\cdVariable{table} - an arbitrary size distribution supplied as a table. This
keyword is case insensitive. The rest of the file contains the parameters
needed to fully define each of those choices. I will now describe these
choices in more detail. It should be noted that at this stage the absolute
normalization of the size distribution is irrelevant; that will be defined in
the refractive index file by the default grain abundance. Each parameter
mentioned below should be entered on a separate line, unless indicated
otherwise. All size parameters should be entered in $\mu$m.

\subsection{ssize}

In this case the size distribution is given by a simple delta function:
\[ n(a) \propto \delta(a - a_0) \]
The only parameter that needs to be supplied is the radius of the grain
$a_0$.

\subsection{ncarb}

This case also describes a single-sized grain, but this time the number
of carbon atoms is the only parameter that is supplied. This is convenient
for defining a PAH molecule of a specific size. This size distribution is
obviously only meaningful when combined with carbonaceous material.

\subsection{power}
In this case the size distribution is given by a simple power law:
\[ n(a) \propto a^\alpha \hspace{2mm} a_0 \leq a \leq a_1 \]
Hence this distribution needs three parameters, which need to be supplied
in the order $a_0$, $a_1$, $\alpha$.

\subsection{exp1, exp2, exp3}
In this case the size distribution is given by a power law with a first-,
second-, or third-order exponential cutoff:
\[ n(a) \propto a^\alpha F(a;\beta) C_l(a;a_l,\sigma_l) C_u(a;a_u,\sigma_u)
\hspace{2mm} a_0 \leq a \leq a_1. \] The function $F$ is included to give
extra curvature in the power-law region, the functions $C_l$ and $C_u$ define
the cutoff of the distribution below $a_l$ and above $a_u$. These functions
are defined as follows:
\[ F(a;\beta) = \left\{
	\begin{array}{l@{\hspace*{4mm}}l}
	    (1 - \beta a)^{-1} & \mbox{if } \beta < 0 \\
	     1 & \mbox{if } \beta = 0 \\
	    (1 + \beta a)      & \mbox{if } \beta > 0
        \end{array}
    \right.
\]
\[ C_l(a;a_l,\sigma_l) = \left\{
        \begin{array}{l@{\hspace*{4mm}}l}
	    \exp \left[ - \left( \frac{a_l - a}{\sigma_l} \right)^n \right] & \mbox{if } a < a_l \\
	    1 & \mbox{if } a \geq a_l
        \end{array}
     \right.
\]
\[ C_u(a;a_u,\sigma_u) = \left\{
        \begin{array}{l@{\hspace*{4mm}}l}
	    1 & \mbox{if } a \leq a_u \\
	    \exp \left[ - \left( \frac{a - a_u}{\sigma_u} \right)^n \right] & \mbox{if } a > a_u
        \end{array}
     \right.
\]
The values of $\sigma_l$ or $\sigma_u$ may be set to zero, in which case a
straight cutoff in the size distribution will be used. Note that when $\beta$,
$\sigma_l$, and $\sigma_u$ are all set to zero, this size distribution
degenerates to the simple power law discussed above. The parameters need to be
supplied in the following order: $a_l$, $a_u$, $\alpha$, $\beta$, $\sigma_l$,
$\sigma_u$, $a_0$, $a_1$. The value of $n$ is determined by the keyword used:
$n=1$ for \cdVariable{exp1}, etc. The parameter $\beta$ should be entered
in $\mu$m$^{-1}$. Note that this size distribution extends
infinitely beyond $a_l$ and $a_u$, so additional cutoffs at $a_0$ and $a_1$
are needed. Either of these values may be set to zero, in which case Cloudy
will calculate a safe default value such that only a negligible amount of mass
is contained in the tail beyond that limit.
\subsection{normal}
In this case the size distribution is given by a Gaussian distribution in $a$:
\[ n(a) \propto \frac{1}{a} \exp \left( -\frac{1}{2} \left[ \frac{a -
      a_c}{\sigma} \right]^2 \right) \hspace{2mm} a_0 \leq a \leq a_1 \] The
parameters for this distribution need to be supplied in the order: $a_c$,
$\sigma$, $a_0$, $a_1$. As was discussed in the previous section, the values
of $a_0$ or $a_1$ may be set to zero in which case Cloudy will calculate a
safe default.
\subsection{lognormal}
This case is completely analogous to the \cdVariable{normal} case discussed
above, except that the distribution is now given by:
\[ n(a) \propto \frac{1}{a} \exp \left( -\frac{1}{2} \left[
    \frac{\ln\{a/a_c\}}{\sigma} \right]^2 \right) \hspace{2mm} a_0 \leq a \leq
a_1 \]
\subsection{table}
\label{sdtable}
This option allows the user to define an arbitrary size distribution in the
form of a table of $a^4 n(a)$ as a function of $a$. First values for the lower
and upper size limit $a_0$ and $a_1$ should be supplied. These values need not
coincide with the lower and upper size limit of the table, although the range
of the table should be at least that large (no extrapolation will be
performed). Next the number of data pairs $n$ in the table should be supplied,
followed by $n$ lines each containing two numbers: $a$ (in $\mu$m) and $a^4
n(a)$ (in arbitrary units). Note that the values for $a$ in the table must be
strictly monotonically increasing.
