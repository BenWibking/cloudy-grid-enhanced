\chapter{OPTICAL DEPTHS AND RADIATIVE TRANSFER}
% !TEX root = hazy1.tex

\section{Overview}

Line transfer is relatively unimportant in
low-density clouds such as \hii\ regions
and planetary nebulae.
Radiative transfer can be important in other
environments,
such as nova shells and the broad-line region of active nuclei,
where excited states of hydrogen have significant
populations and subordinate
lines become optically thick.
In other cases grains are present and all
lines can be absorbed by background opacity.
All radiative transfer effects
are included in the treatment of line formation,
including line thermalization,
destruction by background opacities, pumping by the incident continuum,
and escape from the cloud.

It is necessary to iterate upon the solution if emission lines are
optically thick since total optical depths are not known on the first
iteration.
The default is for a single iteration.
This is
often adequate for low-density nebulae such as planetary nebulae or \hii\ regions.
A second iteration is sometimes enough to establish a fairly
accurate line optical depth scale for many resonance transitions.
Further
iterations are usually needed when subordinate lines are also optically
thick.
The \cdCommand{iterate to convergence} command
will iterate until the optical-depth scale is well defined.

Line-radiation pressure cannot be computed accurately until the total
line optical depths are known,
so this quantity is only meaningful after
the first iteration.
\Cloudy\ will stop if the internal radiation pressure
exceeds half of the surface gas pressure in a constant-pressure
model since
such a geometry is unstable unless it is self-gravitating.
The radiation
pressure is not allowed to exceed half the gas pressure on the initial
iterations of a multi-iteration constant-pressure model.
This is to prevent
the calculation from stopping when the optical depth scale is
not yet well converged.

The following sections outline various commands that affect the
radiative transfer.

\section{Case A [options]}

This has the same options as the Case~B command but sets the \la\ optical
depth to a very small value by default.
This does not turn off induced
processes, which are normally ignored when Case A is assumed.
You would
use the \cdCommand{no induced processes} command to do that.

\section{Case~B [tau ly alpha = 9; options]}
\label{sec:CaseBCommand}

This command is used to check the line emission from  hydrogen-like and
helium-like species in the Case~B limit (AGN3 section 4.2).
This command \emph{should
not} be used in any model that is supposed to represent a real physical
environment.
It is intended only to provide an easy way to check predictions
of the code against simple, more limited, calculations.
In particular,
when this is used the Lyman-line optical depths will be made artificially
large.
This may affect the ionization and temperature of the gas.

With no options, this command sets the inward-looking optical depth of
\la\ for all
atoms and ions of the H- and He-like iso-electronic sequences to
$10^5$ so
that even a one-zone model will be close to
Case~B.\footnote{Before version
96 the default optical depth was $10^9$.  This caused
extreme L$\alpha $ behavior in a grain-free \hii\ region.  The lower value is a better
estimate of the physics that occurs in an actual \hii\ region.}
This optical depth does not depend on the abundance of the element, or even
the presence of the species.
The optional number
is the log of the L$\alpha $ optical depth.
One-sided escape probabilities are
used so the total escape probability is simply that for the inward direction.
In keeping with the Case~B approximation the \cdCommand{Case~B} command suppresses
excited-state line optical depths.

{\bf The Hummer option.}
The species include all collisional processes.
Case~B does not define
the population of the ground or first excited state so a true comparison
with Case~B results should have collisions from these levels turned off.
This is done with the Hummer and Storey option (with the key \cdCommand{Hummer}), to allow
comparison with their 1987 and 1995 papers.
Collisions from the ground
and first excited states \emph{are} included if this second option is not specified.
Collisions between levels with $n\ge  3$ \emph{are} included unless the
\cdCommand{database H-like collision off}
or \cdCommand{database He-like collision off} commands are given.  Collisions
between the $2s$ and $2p$ levels are always included unless the
\cdCommand{database H-like collisions l-mixing off} command is given.
Similarly, this option uses the \citet{Percival1978} $n$-changing collisions and the
classical \citet{PengellySeaton1964} theory for $l$-changing collisions.

In the case of the He-like isoelectronic sequence
the \cdCommand{Case~B} command
sets the optical depths in the singlet Lyman lines to a large value.
The
Hummer \& Storey option has no effect on the He-like sequence.

The \cdCommand{no Pdest} option turns off destruction of
Lyman lines by background opacity.

There are several side effects of this command that may after the spectrum
or physical conditions in unexpected ways.
The large L$\alpha $ optical depth will
often result in an especially strong radiation field within this line.
This affects the gas through photoionization of excited metastable states
of H and He and of those elements
with a small enough ionization potential.
The \cdCommand{no photoionization} option on the \cdCommand{Case~B}
command tells the code not to
include photoionization from excited states of \hO.
But these strong diffuse fields will also strongly affect the level
of ionization of the gas, making the resulting ionization equilibrium a
fiction.
Optically thin gas is actually described by Case C
(\citealp{Ferland1999}, \citealp{LuridianaEtAl09} and
AGN3 Section 11.4) where continuum pumping enhances Balmer lines.
The large
Lyman-line optical depths that result from the \cdCommand{Case~B}
command will prevent
continuum resonant pumping of the atom.  Beware.

\section{Case C [options]}

This has the same options as the Case~B command but sets the
L$\alpha$ optical depth to a very small value by default.
Case C is described by \citet{Ferland1999} and
\citet{LuridianaEtAl09}.

\section{Diffuse fields [outward, OTS]}

This specifies which method is to be used to transfer the diffuse fields,
the emission from gas within the computed structure.
The options are \cdCommand{outward only} and \cdCommand{OTS}.

The \cdCommand{OTS} option takes into account optical depths
in both the inward and outward directions.
The \cdCommand{OTS} option has a \cdCommand{SIMPLE} option which will
do a very simple OTS approximation without taking optical depths into
account.
All diffuse fields with energies capable of ionizing hydrogen
are assumed to do so, and those with smaller energies freely escape.
This is intended as a debugging tool.

If \cdCommand{outward} is chosen then the code will check for a number.
This determines which of the many forms of the outward-only
approximation (\citealp{Tarter1967}) is used.
The default\footnote{OTS was the default in version 86 and before.} is 2.  This is intended for testing the code.

This choice does not strongly affect the predicted emission-line spectrum
but it does change the temperature at the illuminated face of the cloud.

\section{Double optical depths}

On second and later iterations the code uses the total optical depths
of the computed structure to find the outwardly-directed radiation field.
This command doubles the total optical depth so that the shielded face of
the cloud becomes the mid-plane of a structure that is twice as thick as
the computed cloud.

This original purpose of this command was to simulate a geometry in which
ionizing radiation strikes the plane-parallel cloud from both sides.
Examples are a L$\alpha$ forest cloud or the diffuse ISM.
The total line and
continuum optical depths are set to twice the computed optical depth at
the end of the iteration.
The computed model is then one half of the cloud
and the other half of the cloud is assumed to be a mirror image
of the first half.
Doubling the total line and continuum optical depths at the end of
the iteration is the \emph{only} effect of this command.
Physical quantities such
as the physical thickness, column densities, or line emission
\emph{are not} affected.

This approximation makes sense if the cloud is optically thick in lines
but optically thin (or nearly so) in continua.
Lines such as the L$\alpha $
transitions of He I and He II can be important sources
of ionizing radiation.
Their transport will be handled correctly in this limit when this command
is used.
Continuum transport out of the cloud will also be treated
correctly, but attenuation of the incident continuum will
\emph{not} be if the
cloud is optically thick in the continuum.

The second use of this command is when the outer edge of a computed
structure is not the other edge of the cloud.
A typical PDR calculation
is an example.
The calculation starts at the illuminated face and continues
until the gas becomes cool and molecular.  The stopping point often does
not correspond to the outer boundary of the molecular cloud, but rather
is a point that is ``deep enough'' for a given study.  The optical depths
are always computed self-consistently.  On second and later iterations the
total optical depths are normally those of the computed structure.  Near
the shielded face the outward optical depths will be small and radiation
will freely escape in the outward direction.  The gas temperature may fall
dramatically due to the enhanced cooling resulting from the free escape
of line photons.
In real PDRs considerable neutral or molecular material
probably extends beyond the stopping point so that line photons do not freely escape.
The shielding effects of this unmodeled extra material can be
included with this command.
Then, the shielded face of the cloud will
correspond to the mid-plane of the overall structure and lines will not
artificially radiate freely into the outer (unmodeled) hemisphere.

\section{Iterate [2 times, to convergence]}
\label{sec:IterateCommand}

This specifies the number of iterations to be performed.
The default
is a single iteration, a single pass through the model.
At least a second
iteration should be performed in order to establish the correct total optical
depth scale when line transfer or radiation pressure is important.
Two
iterations are sometimes sufficient and will be done if no numbers are
entered on the command line.
A comment will be printed after the last
iteration if the total optical depth scale has not converged and further
iterations are needed.

\subsection{Number of iterations}

There is a slight inconsistency in how the code counts the number of
iterations.
The way it functions in practice is what makes the most sense
to me.

The word \emph{iterate} is from Latin for ``again.''
So the true number of
``agains'' should be one less than the total number of calculations of the
cloud structure.
When the \cdCommand{iterate} command is not entered there is one
calculation of the structure and so formally no iterations.
 If any one
of the following commands is entered:
\begin{verbatim}
iterate
iterate 0
iterate 1
iterate 2
\end{verbatim}
then exactly two calculations of the structure will be done.
If the number
on the line is two or greater, then the number will be the total number
of calculations of the structure.

\subsection{Iterate to convergence [max =7, error =.05, all] }

This is a special form of the \cdCommand{iterate} command
in which the code will
continue to iterate until the optical depths have converged or a limit to
the number of iterations has been reached.
The optional first number on
the line is the maximum number of iterations to perform with a default of
10.  The second optional number is the convergence criterion.
The default
is for relative optical depths to have changed by less than \autocv\ between
the last two iterations.
The optional numbers may be omitted from right
to left.
If all transitions are optically thin then only a second iteration
is performed.

If the \cdCommand{all} option is given, the convergence test will be
made on all transitions, rather than a selected set.

Section \ref{sec:ConvergenceProblems} discusses some reasons the
simulation may not converge.  
There are likely to be convergence problems if the outer edge of the cloud
is set by the lowest allowed temperature rather than an outer radius of column density.

\subsection{Convergence problems}
\label{sec:ConvergenceProblems}

The code generally will not converge if it has not done so within ten
or so iterations.
The most common reason for convergence problems is that
the outer edge of the cloud changes from iteration to iteration.
To prevent this from happening it is important to understand why the
calculation stopped (Section \ref{sec:StoppingCriteria}).

Convergence problems often happen if the outer edge of the cloud is set by
the lowest allowed temperature, as set with the
\cdCommand{stop temperature} command (Section \ref{sec:CommandStopTemperature}),
or the default lowest allowed temperature.
The temperature at the shielded face of the cloud can be affected by the total optical depths.
As a result the point where the lowest temperature is reached can change from
iteration to iteration, so the column densities and optical depths also change,
independently of the radiative transfer solution.
It will not be possible to converge the optical depths since so much is changing.
The code will generate a caution if the solution is not converged after
the limit to the number of iterations is reached.
In that case it will also check if the calculation stopped due to the low-temperature limit
being reached, and suggest changing the stopping criteria in that case.

Another common reason for convergence problems is that
the specified column density or thickness causes the simulation to end
within a prominent ionization front.
In this case very small changes in the
physical conditions result in large changes in the optical depths.  

These
are physical, not numerical, problems.
To prevent them, understand what sets the outer edge of the cloud. 
The code should not have convergence
problems if the outer edge is determined by the outer radius, total column density,
or an equivalent, such as $A_V$.

\section{No scattering opacity}
\label{sec:CommandNoScatteringOpacity}

This turns off several pure scattering opacities.  These include
scattering by grains, electron scattering, and the extreme damping wings
of Lyman lines (Rayleigh scattering).
When scattering opacity is included
and an open geometry is computed the scattering opacity is assumed to
attenuate the incident radiation field as
$\left( {1 + 0.5\,\tau _{scat} } \right)^{ - 1} $
rather than $\exp \left( { - \tau } \right)$ (\citealp{Schuster1905}).

Scattering can be neglected in a spherical geometry with gas fully
covering the source of ionizing radiation.
Scattered photons are not really
lost but continue to diffuse out with (perhaps) a slight shift in energy.
Electron scattering is generally the most important scattering opacity in
a grain-free mixture.
If $ \tau _{scat}  \le 1$
then it is reasonable to consider electron scattering as a heating and
cooling process but not as an absorption mechanism if the energy shifts
are not large (i.e., $ h\nu \ll mc^2$) and the geometry is spherical
(this is not correct for $\gamma$-ray energies,
of course).
\Cloudy\ is not now designed to work in environments that are
quite Compton thick, but should work well for clouds where the electron
scattering optical depths are less than or of order unity.

When this command
is entered scattering processes such as Compton energy exchange 
and grain scattering are still included
as heating, cooling, and ionization processes, but not as extinction sources.
(Thermal and ionization effects of Compton scattering are turned off with
the \cdCommand{no Compton} command).
The \cdCommand{no scattering opacity} command is automatically
generated when \cdCommand{sphere} (see section \ref{sec:CommandSphere}) is specified.

\section{Turbulence = 100 km/s [log, dissipate]}

This enters a microturbulent velocity $u_{turb}$.
The velocity is given in
km s$^{-1}$ on the command line although the code works with
cm s$^{-1}$ internally. The turbulent line width $u_{turb}$ is zero by default,
but the value that is entered must be $u_{turb} \ge 0$.
If the optional keyword \cdCommand{log} appears then the number
is interpreted as the log of the turbulence. Alternatively, you can
also enter the keyword \cdCommand{equipartition} which is discussed
further below.

Turbulent pressure is included in the equation of state when the total
pressure is computed.\footnote{Turbulence was not included as a pressure term in versions 06.02
and before.}
The \cdCommand{no pressure}
option on this command says not to include turbulent pressure in the total
pressure.

Turbulence affects the shielding and pumping of lines.
Fluorescent
excitation of lines becomes increasingly important for larger turbulent
line widths since a larger part of the continuum can be absorbed by a line.
Line pumping is included as a general excitation mechanism for all lines
using the formalism outlined by \citet{Ferland1992} and described further in
a section of Part 3.
The line-center optical depth varies inversely with
the line width velocity $u$ so the effects of line optical depths
and trapping become smaller with increasing line width.
Larger $u$ inhibits self-shielding.

\subsection{Definitions }

The Doppler width of any line that is broadened by both thermal motions
and turbulence that is given by a Gaussian is given by the quadratic sums
of the thermal and turbulent parts,
\begin{equation}
u = \sqrt {u_{th}^2  + u_{turb}^2 }  = b = \sqrt 2 \sigma\;
 [\cmps ].
\end{equation}
Here $b$ is the Doppler parameter used in much of the UV absorption-line
literature, $\sigma $ is the standard deviation
(often called the dispersion) for
a normal distribution, and $\sigma^2$ is the variance.
For a thermal distribution
of motions the average velocity along the line of sight
(\citealp{Mihalas1978}, equation 9-35, page 250)
and the most probable speed (\citealp{Novotny1973}, p 122)
of a particle with mass $m$ are both given~by
\begin{equation}
u_{th}  = \sqrt {2kT/m}\;
[\cmps ].
\end{equation}
This corresponds to $16.3 \sqrt{ t_4/m_{AMU}}\ {\rm km\ s}^{-1}$ for pure thermal motions.
This command sets $u_{turb}$.
Note that with these definitions the full-width half-maximum
(FWHM) of a line is equal to
\begin{equation}
u\left( {FWHM} \right) = u\sqrt {4\ln 2}\;
[\cmps ].
\end{equation}

\subsection{Turbulent pressure}

For an ideal gas the thermal pressure is $P = nkT$
while the energy density is $1/2 nkT$
per degree of freedom, so for a monatomic gas is $U = 3/2 nkT$.
Turbulent motions add pressure and energy terms that are analogous
to thermal motions if the turbulence has a Gaussian distribution.
Note
that the turbulent velocities may be organized, or lined up in certain
directions, if the gas is magnetically controlled.
This presents a
complication that is ignored.

\citet{HeilesTroland2005} discuss both turbulence and magnetic fields.
Their equation 34 gives the turbulent energy density in terms of the
one-dimensional turbulent velocity dispersion $\Delta V_{turb,1D}^2 $
(or standard deviation).
Note that we write velocities in terms of $u$
with the relationship $u^2  = 2\Delta V_{turb,1D}^2 $.
Their energy density can be rewritten as a pressure:
\begin{equation}
\begin{array}{ccl}
 P_{turb}  = \frac{F}{6}\rho \,u_{turb}^2&  =& 3.9 \times 10^{ - 10} F\left(
{\frac{{n_{tot} }}{{10^5 \;{\mathrm{cm}}^{ - 3} }}} \right)\left(
{\frac{{u_{turb} }}{{1\;{\mathrm{km}}\;{\mathrm{s}}^{ - 1} }}} \right)^2 \quad \left[
{{\mathrm{erg\; cm}}^{{\mathrm{ - 3}}} {\mathrm{;\ dyne\; cm}}^{ - 2}} \right] \\
&  =& 2.8 \times 10^6 \,\,F\,\left( {\frac{{n_{tot} }}{{10^5 \;{\mathrm{cm}}^{
- 3} }}} \right)\left( {\frac{{u_{turb} }}{{1\;{\mathrm{km}}\;{\mathrm{s}}^{ - 1}
}}} \right)^2 \quad \left[ {{\mathrm{cm}}^{ - 3} \;{\mathrm{K}}} \right] \\
 \end{array}
\end{equation}
where $n_{tot}$ is the total hydrogen density,
$u_{turb}$ is the turbulent velocity,
and He/H$ = 0.1$ was assumed.
The term $F$ accounts for how the turbulent
velocity field is ordered (\citealp{HeilesTroland2005}, their equation 34).
$F$ is 2 for turbulent velocities that are perpendicular to
the magnetic field
as in Alfven waves and $F$ is 3 for isotropic turbulent motions.
The default
value of $F = 3$ is changed by entering a new value as a second number on
this command line.

\subsection{Energy dissipation}

The \cdCommand{dissipate} option on the \cdCommand{turbulence}
command provides a way in include
conversion of wave energy into heat (see \citealp{BottorffFerland2002}).
When
the option is used a third number, the log of the scale length for the
dissipation in cm, must appear.
Then the turbulent velocity will have the form
\begin{equation}
u_{turb} (r) = u_{turb} (r_{\mathrm{o}} )\exp ( - \Delta r/r_{scale} )\quad
\mathrm{[cm\, s}^{-1}]% (55)
\end{equation}
where $u_{turb}(r_{\mathrm{o}})$ is the turbulence at the illuminated face
and $\Delta r$ is the depth into the cloud.
The wave mechanical energy is assumed to have been
converted into heat with a local heating rate given by
(\citealp{BottorffFerland2002})
\begin{equation}
G(r) = 3.45 \times 10^{ - 28} 2^{ - 3/2} u_{turb}^3 (r) \quad
\mathrm{[erg\, cm}^{-3} \mathrm{s}^{-1}]% (56)
\end{equation}

\subsection{Equipartition turbulent - magnetic pressures }

The \cdCommand{equipartition} option on the \cdCommand{turbulence} command sets the turbulent
velocity to an equipartition between magnetic and
turbulent energy densities.
That is,
\begin{equation}
\label{eqn:MagneticEquipartition}
P_{turb}  = \frac{F}{6}\rho \,u_{turb}^2  = P_{mag}  = \frac{{B^2 }}{{8\pi
}}\, \mathrm{[erg\, cm}^{-3}].% (54)
\end{equation}
The \cdCommand{magnetic field} command sets $B$.
The turbulent
velocity is determined from the magnetic field
assuming equation \ref{eqn:MagneticEquipartition}.
\citet{HeilesCrutcher2005} argue that the correlation between turbulent and
magnetic pressures, while true on average, does not hold in detail for
specific regions of the ISM. When this option is used, you can optionally add
the parameter $F$ on the command line (the default is $F=3$). The keyword \cdCommand{no pressure} is
supported, but not \cdCommand{dissipate}.

\subsection{Turbulence command heads up!}

\emph{N.B.!}  In the default (non-equipartition) form of the command, the turbulent velocity is the first number
on the command line.
The $F$ parameter is the second number.
The energy-dissipation scale length is the third number and must appear
after $u_{turb}$ and $F$.
These numbers can only be omitted from right to left.
In the equipartition case you can only specify $F$,
which then is the first number on the command line.
The energy-dissipation scale length is not supported in the equipartition case.
The keyword \cdCommand{vary} is only supported in the non-equipartition case,
in which case the turbulent velocity is varied.

The turbulent velocity must be less than the speed of light.
The code will stop if $u_{turb} \geq c$ is specified.

\section{vlaw alpha=-1}

This specifies a turbulent velocity that is a power law in radius.
The number is the power law $\alpha$ on radius.
It must be negative.
The turbulence will be given by
\begin{equation}
u = u_0 \left( {{\raise0.7ex\hbox{$r$} \!\mathord{\left/
 {\vphantom {r {r_0 }}}\right.\kern-\nulldelimiterspace}
\!\lower0.7ex\hbox{${r_0 }$}}} \right)^\alpha
\end{equation}
where $r_0$ in the inner radius and
$u_0$ is the initial turbulence specified with the
\cdCommand{turbulence} command.
