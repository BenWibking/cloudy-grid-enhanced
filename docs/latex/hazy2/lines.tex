\chapter{THE EMISSION LINES}
% !TEX root = hazy2.tex
\label{sec:EmissionLines}

\section{Overview}

The following sections outline the emission lines predicted by \Cloudy.
Before version 90 of the code all lines were listed in the sub-section
immediately following this section.  The code is being modified to bring
all lines into a common line class, as the code moves to C++ and objects.
This chapter will remain incomplete until this work is finished.


\section{The main emission-line printout }

The main emission line printout was briefly described
in the Chapter \cdSectionTitle{OUTPUT}.
This section
goes into more detail.

\cdCommand{Output organization}.  The printed list is sorted into four large groups
of columns, with each large column sub-divided into four smaller sub-columns.
The first sub-column is either the spectroscopic designation of the ion
producing the line or an indication of how the line is formed.  The second
sub-column is the line wavelength, with a 0 to indicate a continuum.  The
third sub-column is the log of the power in the line, in the units given
in the header (erg s$^{-1}$ into either $4\pi$ sr or cm$^{-2}$).  The last sub-column is
the intensity of the line relative to the reference line,
usually H$\beta$ , unless
this is reset with the \cdCommand{normalize} command.

These lines can be printed as a single large column, and can be sorted
by wavelength or intensity.  These options are controlled by the
\cdCommand{print line}
command described in Part I of this document.

\subsection{Intrinsic and emergent line intensities}

See Section \ref{sec:LineIntensitiesDustyCloud}.

\subsection{Units of Line wavelengths}
These can be given in various units.
Numbers ending
in ``A'' are wavelengths in angstrom (\AA).
For instance, H$\beta$ is given by ``H~~1  4861.32A''.
Wavelengths in micron are indicated by ``m'', an example,
the strong [O~III] IR line, is ``O~~3 51.8004m''.

Continua are usually
indicated by a wavelength of zero.

\subsection{Air vs vacuum wavelengths}

The emission line wavelengths follow the convention that vacuum wavelengths
are used for $\lambda_{\rm vac} \le 2000$\,\AA\ and STP air wavelengths are used
for $\lambda_{\rm vac} > 2000$\,\AA.
The \cdCommand{print line vacuum} command tells the code use use vacuum wavelengths throughout.
\cdCommand{Save continuum} output is always reported in vacuum wavelengths to avoid 
a discontinuity at 2000\,\AA.

\subsection{Blocks of lines in main output}

Lines are organized by common origin with a comment, ending in a series
of periods ``\dots'', beginning the section.
As an example, the first
commented block of lines begins with ``general properties\dots''.
The
following subsections give overviews of the lines.

\subsection{General properties\dots}

This mainly summarizes heating and cooling agents for the model.
\begin{description}
\item[H  1 4861 and H  1 1216,] are the total intensities of H$\beta$
and L$\alpha $, as
predicted by the multi-level H atom.
These intensities are the results
of calculations that include all collisional, radiative, and optical
depth effects.

\item[Inci]  The total energy in the incident continuum.
This entry will not be included if the \cdCommand{aperture} command is in effect.

\item[TotH and TotC] give the total heating and cooling.  These will be nearly
equal in equilibrium.

\item[BFH1 and BFHx] are the heating due to photoionization of ground state
and excited state hydrogen respectively.

\item[He1i, 3He1], heating due to ground state He and the triplets.

\item[BFHe and TotM] are the heating due to helium and metal photoionization.

\item[Pair] heating due to pair production.

\item[ComH , ComC],  Compton heating, cooling.

\item[CT H    CT C] charge transfer heating and cooling.

\item[extH   extC]    ``extra'' heating or cooling added to model.

\item[e-e+  511]  The positron line.

\item[Expn], expansion, or adiabatic, cooling

\item[H FB], H radiative recombination cooling

\item[HFBc, HFBc], hydrogen net free-bound cooling and heating

\item[Iind], cooling due to induced recombination of hydrogen

\item[3He2], cooling due to induced recombination of fully ionized helium

\item[Cycn], cyclotron cooling
\end{description}

\subsection{Continua\dots}

These give intensities of various continua.  These are either the total
integrated continuum or the product $\nu F_\nu$ at certain energies.

\subsubsection{Continuum bands}
The file \cdFilename{continuum\_bands.dat} in the data directory
specifies a set of wavelength bands.
The code will integrate over these bands to find the
total radiated luminosity and enter this into the main emission-line stack.
Currently this is a simple sum of the energy emitted between the upper and
lower bounds of the band, and does not take into account the transmission function
of particular instruments.
The \cdFilename{continuum\_bands.dat} file can be edited to change 
the number of bands or their detailed properties.
Table \ref{tab:continuum_bands} lists the bands in the file at the
time of this writing.
The first and second columns give
the label and wavelength as they appear in the printout.
The last column
gives the wavelength range for the integration.
The first wavelength is treated as a label and will always appear in the output
as shown in the table. The last two numbers are interpreted as vacuum wavelengths.
These entries will not be included if the \cdCommand{aperture} command is in effect.
Please consult the file to see its current contents
and feel free to add your own bands.

%copied from ini file 2012 July 20
% commented out code to print this is in cont_createpointers.cpp
% search for string *hazy table*
\begin{table}
\centering
\caption{\label{tab:continuum_bands}Default continuum bands}
\begin{tabular}{lll}
\hline
Label& Wavelength&Wavelength Range\\
\hline
 FIR  & 83.0000m & 40.0000m -- 500.000m\\ 
 TIR  & 1800.00m & 500.000m -- 3100.00m\\ 
 NIRa & 2.85000m & 7000.00A -- 40.0000m\\ 
 NIRb & 3.00000m & 10000.0A -- 5.00000m\\ 
 MIRa & 15.0000m & 5.00000m -- 25.0000m\\ 
 MIRb & 22.5000m & 5.00000m -- 40.0000m\\ 
 NMIR & 21.7500m & 7000.00A -- 40.0000m\\ 
 TFIR & 611.250m & 122.500m -- 1100.00m\\ 
 TALL & 10000.0A & 0.0100000A -- 10000.0m\\ 
 F12  & 12.0000m & 8.50000m -- 15.0000m\\ 
 F25  & 25.0000m & 19.0000m -- 30.0000m\\ 
 F60  & 60.0000m & 40.0000m -- 80.0000m\\ 
 F100 & 100.000m & 83.0000m -- 120.000m\\ 
 MIPS & 24.0000m & 20.8000m -- 26.1000m\\ 
 MIPS & 70.0000m & 61.0000m -- 80.0000m\\ 
 MIPS & 160.000m & 140.000m -- 174.000m\\ 
 IRAC & 3.60000m & 3.16000m -- 3.92000m\\ 
 IRAC & 4.50000m & 4.00000m -- 5.02000m\\ 
 IRAC & 5.80000m & 5.00000m -- 6.40000m\\ 
 IRAC & 8.00000m & 6.50000m -- 9.30000m\\ 
 SPR1 & 250.000m & 212.000m -- 288.000m\\ 
 SPR2 & 350.000m & 297.000m -- 405.000m\\ 
 SPR3 & 500.000m & 414.000m -- 600.000m\\ 
 PAC1 & 70.0000m & 60.0000m -- 82.0000m\\ 
 PAC2 & 100.000m & 84.0000m -- 122.000m\\ 
 PAC3 & 160.000m & 130.000m -- 198.000m\\ 
 PAH  & 3.30000m & 3.25000m -- 3.35000m\\ 
 PAHC & 3.23000m & 3.20000m -- 3.25000m\\ 
 PAHC & 3.37000m & 3.35000m -- 3.40000m\\ 
 PAH  & 6.20000m & 5.90000m -- 6.40000m\\ 
 PAHC & 5.65000m & 5.40000m -- 5.90000m\\ 
 PAH  & 7.90000m & 7.40000m -- 8.40000m\\ 
 PAHC & 6.90000m & 6.40000m -- 7.40000m\\ 
 PAH  & 11.3000m & 11.1000m -- 11.5000m\\ 
 PAHC & 10.9000m & 10.7000m -- 11.1000m\\ 
 PAH  & 11.8000m & 11.6000m -- 12.3000m\\ 
 PAHC & 12.6500m & 12.3000m -- 13.0000m\\ 
 PAH  & 13.3000m & 12.9000m -- 13.7000m\\ 
 PAHC & 14.1000m & 13.7000m -- 14.5000m\\ 
 Bcon & 3640.00A & 911.600A -- 3646.40A\\ 
 Pcon & 5000.00A & 3646.40A -- 8204.40A\\ 
 O3bn & 5007.00A & 4980.00A -- 5100.00A\\ 
 SX   & 25.0000A & 8.26000A -- 41.3000A\\ 
 HX   & 4.70000A & 1.23000A -- 8.26000A\\ 
 ROSA & 12.3980A & 5.16600A -- 123.980A\\ 
\hline
\end{tabular}
\end{table}

\subsubsection{Other continua}

\begin{description}
\item[Bac  3646]  residual flux at head of Balmer
continuum, $\nu F_\nu$. This entry will not be included if the \cdCommand{aperture}
command is in effect.

\item[cout 3646 cref 3646], outward, reflected continuum at peak of Balmer Jump.
These entries will not be included if the \cdCommand{aperture} command is in effect.

\item[thin 3646], residual flux at head of Balmer continuum, optically thin limit.
This entry will not be included if the \cdCommand{aperture} command is in effect.

\item[Inci 4860, Inci 1215], incident continua near \ha\ and \la.
These entries will not be included if the \cdCommand{aperture} command is in effect.

\item[Ba C    0], integrated Balmer continuum

\item[PA C    0], integrated Paschen continuum

\item[HeFF    0], He brems emission

\item[HeFB    0], He recombination cooling

\item[MeFB    0], heavy element recombination cooling

\item[MeFF    0], metal brems emission

\item[ToFF    0], total brems emission

\item[FF x], part of H brems, in x-ray beyond 0.5KeV

\item[eeff], electron - electron brems
\end{description}

\cdTerm{nFnu  122m},
\cdTerm{nInu  122m},
\cdTerm{InwT  122m},
\cdTerm{InwC  122m},
a large list of continua
at selected wavelengths will be printed.
These entries will not be included if the \cdCommand{aperture} command is in effect.
The first is the sum of various continua at the wavelength,
given as $\nu F_\nu$ (see the \cdCommand{set nFnu} command
in Part~1 for a discussion of what is included in the \cdTerm{nFnu} entry).
\cdTerm{nInu} is the transmitted and reflected incident continuum.
\cdTerm{InwT} is the total reflected continuum.
\cdTerm{InwC} is the reflected incident
continuum.

\subsection{Molecules\dots}

\begin{description}

\item[H2dC], is the cooling due to collisional dissociation of \htwo.

\item[H2dH], heating by \htwo\ dissociation by Lyman continuum

\item[H2vH], heating by coll deexcit of vib-excited \htwo\

\item[H2vC], cooling by coll deexcit of vib-excited \htwo\

\item[H2 v], line emission by vib-excited \htwo\

\item[H-FB and H-FF] are the free-bound and free-free continua of the H- ion.

\item[H-CT 6563], H-alpha produce by H- mutual neutralization

\item[H- H]    0, H- heating

\item[H-Hc]    0, H- heating

\item[H2+] and HEH+ are the cooling due to formation of H$_2^+$ and HeH$^+$.

\item[Codh], carbon monoxide photodissociation heating

\item[CO C   12], C12O16 cooling

\item[CO C   13], C13O16 cooling

\end{description}


\subsection{Species Bands\dots}
\label{sec:SpeciesBands}

Bands in which the emission of a certain species is accumulated are reported in
the section of block of lines on the main output headlined with
``\verb+bands....+''.
User-specified bands may be provided with the \cdCommand{save species bands}
command, described in Hazy 1, Section~\ref{Hazy1-sec:SaveSpeciesBands}.
By default, the Fe~{\sc ii} bands defined in \cdFilename{data/FeII\_bands.dat},
if iron is enabled in the model, even if not specified with such a command.

Note that these bands are more fully explained in the output of the command
\cdCommand{save line labels}, see Hazy 1, Section~\ref{Hazy1-sec:SaveLineLabels}.


\subsection{Grains\dots}

Information in this block concerns emission, absorption, heating, and
cooling by any grains included in the calculation.

\begin{description}
\item[GrGH], gas heating by grain photoionization

\item[GrTH], gas heating by thermionic emissions of grains

\item[GrGC], gas cooling by collisions with grains

\item[GraT], This is the total grain heating by all sources, lines, collisions,
incident continuum.  If the grain emission is optically thin limit then
this is equal to the total intensity in grain emission.

\item[GraI], grain heating by incident continuum

\item[GraL 1216], grain heating due to destruction of L$\alpha $

\item[GraC], grain heating due to collisions with gas

\item[GraD], grain heating due to diffuse fields, may also have grain emission
\end{description}

Grain emission is included in the predicted total emitted continuum.
A machine readable form of the continuum can be produced with the \cdCommand{save continuum} command,
also described in Part I of this document.

\subsection{H-like iso-seq\ldots}

This block includes all hydrogen-like isoelectronic species.
The \cdCommand{atom H-like} command,
described in Part 1 of this document, adjusts properties
of this sequence.

\begin{description}
\item[HFFc    0], net free-free cooling, nearly cancels with cooling in lte

\item[HFFh   0], net free-free heating, nearly cancels with cooling in lte

\item[H FF   0], H brems (free-free) cooling

\item[FF H    0], total free-free heating

\item[Clin  912], total collisional cooling due to all hydrogen lines

\item[Hlin  912], total collisional heating due to all hydrogen lines

\item[Cool 1215.67], collisionally excited La cooling

\item[Heat 1215.67], collisionally de-excited La heating

\item[Crst  960], cooling due to n>2 Lyman lines

\item[Hrst  960], heating due to n>2 Lyman lines

\item[Crst 4861.32], cooling due to n>3 Balmer lines

\item[Hrst 4861.32], heating due to n>3 Balmer lines

\item[Crst    0], cooling due to higher Paschen lines

\item[Hrst    0], heating due to higher Paschen lines

\item[LA X 1215.67], L$\alpha $ contribution from suprathermal secondaries from ground

\item[Ind2 1215.67], L$\alpha $ produced by induced two photon

\item[Pump 4861.32], H$\beta$ produced by continuum pumping in optically thin ld limit

\item[CION    0], net col ionz-3 body heat collision ionization cooling of
hydrogen

\item[3bHt    0], heating due to 3-body recombination

\item[Strk 1215.67], Stark broadening component of line

\item[Dest 1215.67], part of line destroyed by background opacities

\item[Fe 2 1215.67], part of L$\alpha $ absorbed by Fe II

\item[Q(H) 4861.32] is the intensity of H$\beta$ predicted from the total number of
ionizing photons, Q(H$^0$), assuming that each hydrogen-ionizing photon produces
one hydrogen atom recombination. This entry will not be included if the \cdCommand{aperture}
command is in effect.

\item[Q(H) 1215.67] indicates the L$\alpha $ intensity produced if each hydrogen ionizing
photon results in one \la\  photon in the high density limit (i.e., no
two-photon emission). This entry will not be included if the \cdCommand{aperture}
command is in effect.

\item[CaBo 4861.32] These are the ``old'' case B predictions, as printed in versions
90 and before of the code.
\end{description}

\cdTerm{Ca B 6562.80A} The entries starting with ``Ca~B''
are the Case B intensities
computed from the actual model ionization and temperature structure, but
assuming that H$\beta$ emits with its Case~B emissivity.

Next the predicted intensities of all lines of the hydrogenic
iso-electronic sequence are given.
The lines have labels that identify
the species and stage of ionization,
such as ``H~~1'', ``He~2'', ``Li~3'', ``C~~6'', etc.
The entries with double the wavelength of the Ly$\alpha$ line are the total intensities
of the $2s-1s$ two-photon emission. The M1 line between these two levels is also reported
and has a wavelength that is very close to the $2p_{1/2}-1s_{1/2}$ line. To discriminate it from
the latter, ``M1'' is added to the line label. For light elements the two-photon emission
is the dominant transition between these two levels, but for heavy elements the M1 line dominates.

\subsection{He iso-sequence\dots }

Atoms and ions of the helium-like iso-electronic sequence are treated
as multi-level atoms.  All species and stages of ionization are specified
by labels like ``He~1'', ``Li~2'', ``C~~5'', etc.
The wavelength of the two-photon continuum $1s.2s\ ^1S_0-1s^2\ ^1S_0$ is reported
as twice the wavelength derived from the energy difference between the
two levels. The $2\;^3P$ term is resolved into three levels.
Emission of
each line of the $2\;^3S - 2\;^3P$ and $1\;^1S - 2\;^3P$ multiplets
is predicted along with the sum of
the multiplets with label ``Blnd''.
The \cdCommand{atom He-like} command, described in Part 1 of
this document, adjusts properties of this sequence.
Further details are
given in \citet{Bauman2005}, \citet{Porter2005}, and \citet{PorterFerland2007}.

\subsection{Recombination\dots}

These are a set of heavy-element recombination lines that are predicted
in the low-density limit assuming that the transitions are optically thin.
This consists of all recombination lines of C, N., and O, with coefficients
taken from \citet{Nussbaumer1984} and \citet{Pequignot1991}.

These predictions are for optically thin pure recombination.
These should
be accurate for planetary nebulae and H II regions.
They will not be
accurate for dense environments where optical depths and collisional effects
come into play.
These are only included in the output if the electron
density is less than $10^8 \pcc$,
a rough upper limit to the range of validity
in the original calculations of the coefficients.

There are several instances where more than one line of an ion will have
the same wavelength.

\subsection{Level 2 lines\dots }

These are resonance lines that use Opacity Project wavelengths, which
are generally accurate to about 10\%.
These lines have g-bar collision
strengths, which are not very accurate at all.

\section{The transferred lines}
\label{sec:TransferredLines}

\subsection{Save line data output}

The group of ``transferred lines'' includes all those that have been
moved to the \cdFilename{EmLine} class.

In older versions of this document a large list of emission lines
appeared here.
This list is now far too large to include here.
Rather,
the list can be generated by executing the code with the command \cdCommand{save line data ``filename.txt''} included.
This will create a file that includes the
full set of lines that are predicted.
Note that the lines that are output
are only those that exist when the code is run.
It is possible to make
many of the model atoms and molecules as large or small as you like,
and
the actual lines that exist when the \cdCommand{save} command
is entered will be output.
The test case \cdFilename{func\_lines.in} in the test suite
includes this save command and generates a list of all transferred
lines in the file \cdFilename{func\_lines.lis}.

To generate a line list, set up a calculation with the atoms set to
whatever size is desired (see the \cdCommand{atom} command in Part I).  Then execute
this script with the \cdCommand{save line data} command included (described in Part I).
The save output will include the line list.
This will include the level 2, H and He iso-sequence,
\htwo, CO, and recombination lines,
but not the scalar forbidden lines.
These are described in a list following this subsection.

This contains several groups of lines.
All quantities were evaluated at $10^4 \K$.
The description of the command in Part I of this document explains
how to evaluate the quantities at other temperatures.

The ion is the first column of the table.
This is in a uniform format,
beginning with the two character element symbol and followed by an integer
indicating the level of ionization.
``C~~2'' is C$^+$ or C~II.
This is
followed by the wavelength used to identify the line in the
printout.
The third column, with the label ``WL'', is the correct wavelength
of the line, with units of micron (``m''), angstrom (``A''),
or centimeter (``c'').
The remaining columns give the statistical weights of the lower and upper
levels, the product of the statistical weight and the oscillator strength,
and then the transition probability.

The last column is the electron collision strength.
Usually these collision strengths
are for only the indicated transition, although in some cases (the Be
sequence) the value is for the entire multiplet.

\subsection{Output produced for the transferred lines}

Because the lines have a common format within their storage vectors,
the output has a common format too.
Generally only the total intensity
of the transition, the result of the solution of a multi-level atom with
all processes included, is printed.
The approach used to compute the level
populations is described in Part II of \Hazy,
and includes continuum pumping,
destruction by background opacities, and trapping.

The total intensity of the transition is printed in a form like ``C~~2 1335'', with the spectroscopic identification given by the first part,
as found in the first column of the table, and the wavelength as indicated
by the number in the second column of the table.

In a few cases (for instance, the \ion{C}{4} $\lambda\lambda$1548, 1551 doublet), a total
intensity is also derived.
In these cases the label ``Blnd'' will appear
together with an average wavelength (1549 in this case).
These lines are
all explicitly shown in a following section.

It is possible to break out various contributors to the lines with options
on the \cdCommand{print line} command, described in Part I of this document and in the
following.
These contributors are printed following the total intensity.

\cdCommand{print line heating}  An emission line will heat
rather than cool the gas
if it is radiatively excited but collisionally de-excited.
The print out
will include this agent, with the label ``Heat'',
when this command is given.

\cdCommand{print line collisions}  The collisional contribution
to the lines will
be printed, with the label ``Coll''.

\cdCommand{print line pump}  The contribution to the total line, produced by continuum
pumping, is printed with the label ``Pump''.   What is observed?  Whether
or not this is a net emission process contributing to the observed line
intensity depends on the geometry, mainly whether or not continuum source
is in the beam. At some velocities within the line profile this can be a
net emission process, due to absorption at other velocities.   If the
continuum source is in the beam and gas covers it, this is not a net emission
process, since photons are conserved.

\cdCommand{print line inward}  The inwardly directed part of the total emission is
printed with the label ``Inwd''.  This can be greater than half of the line
intensity if the line is optically thick since these lines tend to be
radiated from the hotter illuminated face of the cloud.

\cdCommand{print line optical depths}  At the end of the calculation the optical
depths for all optically thick lines will be printed.
This is not done
by default since it can be quite long.

\section{Line identifications}

The file \cdFilename{line\_labels.txt} in the docs directory of the distribution lists
line identifications and a brief description of its origin.

\section{Hydrogen recombination lines}

Table \ref{tab:HydrogenEmissionLines} gives the strongest lines
of the lowest series.
All lines have the label ``H~~1''.
The wavelength column gives
the string as it appears in the printout.  ``m'' indicates a wavelength
in $\mu$m and A in angstrom.
The Case B intensity is taken from the
\cdFilename{limit\_caseb\_h\_hs87} test case.

Some lines may have the same wavelength if the default line precision is used.
The \cdCommand{set line precision} command can be used 
to increase the number of significant figures in the
line wavelength, which should make identification easier.

% list update 17 02 18 gjf to fix error in Brackett line n'-n and improve wl
\begin{table}
\centering
\caption{\label{tab:HydrogenEmissionLines}Hydrogen emission lines}
\begin{tabular}{llll}
\hline
Series& $n'\to n$& Wavelength& $I$(Case B)\\
\hline
Lyman& 2--1& 1215.68A& - \\
Balmer H$\alpha$& 3--2& 6562.85A     &2.8463\\
H$\beta$& 4--2& 4861.36A     &1.0000\\
H$\gamma$& 5--2& 4340.49A     &0.4692\\
H$\delta$& 6--2& 4101.76A     &0.2596\\
Paschen P$\alpha$& 4--3& 1.87511m   & 0.3319\\
P$\beta$& 5--3& 1.28181m    &0.1617\\
P$\gamma$& 6--3& 1.09381m    &0.0899\\
P$\delta$& 7--3& 1.00494m    &0.0553\\
Brackett Br$\alpha$& 5--4& 4.05116m &    0.0778\\
Br$\beta$& 6--4& 2.62515m&     0.0445\\
Br$\gamma$& 7--4& 2.16553m&     0.0274\\
Br$\delta$& 8--4& 1.94456m&     0.0180\\
Pfund& 6--5& 7.45781m& 0.0246\\
Humphreys& 7--6& 12.3685m& 0.0098\\
& 8--7& 19.0566m& 0.0042\\
& 9--8& 27.7957m& 0.0020\\
\hline
\end{tabular}
\end{table}

\section{Molecular hydrogen lines}

Table \ref{tab:MolecularHydrogenLines} gives some of the stronger
or more frequently observed \htwo\ lines.
These are only predicted when the large model \htwo\ molecule is
included with the \cdCommand{atom H2} command.
The wavelength column gives the string
as it appears in the printout.  ``m'' indicated a wavelength in $\mu$m.
All lines have the label ``H2~~''.

Many lines will have the same wavelength if the default line precision is used.
The \cdCommand{set line precision} command can be used 
to increase the number of significant figures in the
line wavelength, which should make identification easier.

\begin{table}
\centering
\caption{\label{tab:MolecularHydrogenLines}Molecular hydrogen emission lines}
\begin{tabular}{lllll}
\hline
Transition& $v_{hi},J_{hi}$& $v_{lo},J_{lo}$& $\lambda$(label)& $\chi$(hi,
K)\\
\hline
0-0 S(0)& 0,2 & 0,0& 28.2130m & 509.8\\
0-0 S(1)& 0,3 & 0,1& 17.0300m & 1015.1\\
0-0 S(2)& 0,4 & 0,2& 12.2752m & 1681.6\\
0-0 S(3)& 0,5 & 0,3& 9.66228m & 2503.8\\
0-0 S(4)& 0,6 & 0,4& 8.02362m & 3474.3\\
0-0 S(5)& 0,7 & 0,5& 6.90725m & 4586.2\\
0-0 S(6)& 0,8 & 0,6& 6.10718m & 5829.5\\
1-0 O(2)& 1,0 & 0,2& 2.62608m & 5986.9\\
1-0 Q(1)& 1,1 & 0,1& 2.40594m & 6149.0\\
1-0 O(3)& 1,1 & 0,3& 2.80176m & 6149.0\\
1-0 S(0)& 1,2 & 0,0& 2.22269m & 6471.4\\
1-0 Q(2)& 1,2 & 0,2& 2.41307m & 6471.4\\
1-0 O(4)& 1,2 & 0,4& 3.00305m & 6471.4\\
1-0 S(1)& 1,3 & 0,1& 2.12125m & 6951.3\\
1-0 Q(3)& 1,3 & 0,3& 2.42307m & 6951.3\\
1-0 O(5)& 1,3 & 0,5& 3.23411m & 6951.3\\
0-0 S(7)& 0,9 & 0,7& 5.50996m & 7196.7\\
1-0 S(2)& 1,4 & 0,2& 2.03320m & 7584.3\\
1-0 Q(4)& 1,4 & 0,4& 2.43683m & 7584.3\\
1-0 O(6)& 1,4 & 0,6& 3.50043m & 7584.3\\
1-0 S(3)& 1,5 & 0,3& 1.95702m & 8365.0\\
1-0 Q(5)& 1,5 & 0,5& 2.45408m & 8365.0\\
1-0 O(7)& 1,5 & 0,7& 3.80648m & 8365.0\\
0-0 S(8)& 0,10& 0,8& 5.05148m & 8677.0\\
1-0 S(4)& 1,6 & 0,4& 1.89145m & 9286.3\\
1-0 Q(6)& 1,6 & 0,6& 2.47510m & 9286.3\\
0-0 S(9)& 0,11& 0,9& 4.69342m & 10261\\
1-0 S(5)& 1,7 & 0,5& 1.83529m & 10341\\
1-0 S(6)& 1,8 & 0,6& 1.78746m & 11521\\
1-0 S(7)& 1,9 & 0,7& 1.74760m & 12817\\
\hline
\end{tabular}
\end{table}

The following is an example which predicts the emissivity of the \htwo\ 2.121 1-0 S(1) \micron\ line.
\begin{verbatim}
save line emissivity ``lines.ems''
H2   2.121m
end of lines
\end{verbatim}

A list of all H2 lines with their labels and excitation energies can
be generated with the command \cdCommand{save H2 lines}.
\href{http://jach.hawaii.edu/UKIRT/astronomy/calib/spec\_cal/h2\_s.html}{The Joint Astronomy Centre} in Hawaii give a
summary of \htwo\ lines.

\section{CO lines}

Rotation transitions within the ground vibration level are included.  
These are listed in Table \ref{tab:COLines} which gives the symbol
``m'' to indicate $\micron$.

\begin{table}
\centering
\caption{\label{tab:COLines}CO emission lines}
\begin{tabular}{ll}
\hline
Label & wavelength \\
\hline
CO      & 2600.05m \\
CO      & 1300.05m \\
CO      & 866.727m \\
CO  	& 650.074m \\
CO  	& 520.089m \\
CO  	& 433.438m \\
CO  	& 371.549m \\
CO  	& 325.137m \\
CO  	& 289.041m \\
CO  	& 260.169m \\
CO  	& 236.549m \\
CO  	& 216.868m \\
CO  	& 200.218m \\
CO  	& 185.949m \\
CO  	& 173.584m \\
CO  	& 162.767m \\
CO  	& 153.225m \\
CO  	& 144.745m \\
CO  	& 137.159m \\
CO  	& 130.333m \\
CO  	& 124.160m \\
CO  	& 118.548m \\
CO  	& 113.427m \\
CO  	& 108.733m \\
CO  	& 104.416m \\
CO  	& 100.433m \\
CO  	& 96.7462m \\
CO  	& 93.3237m \\
CO  	& 90.1384m \\
CO  	& 87.1666m \\
CO  	& 84.3877m \\
CO  	& 81.7835m \\
CO  	& 79.3383m \\
CO  	& 77.0375m \\
CO  	& 74.8697m \\
CO  	& 72.8231m \\
CO  	& 70.8878m \\
CO  	& 69.0557m \\
CO  	& 67.3181m \\
CO  	& 65.6684m \\
\hline
\end{tabular}
\end{table}

